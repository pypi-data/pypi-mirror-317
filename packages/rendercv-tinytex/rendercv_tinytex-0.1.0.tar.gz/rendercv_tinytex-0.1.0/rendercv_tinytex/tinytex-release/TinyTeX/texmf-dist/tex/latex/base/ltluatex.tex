%%
%% This is file `ltluatex.tex',
%% generated with the docstrip utility.
%%
%% The original source files were:
%%
%% ltluatex.dtx  (with options: `tex,plain')
%% 
%% This is a generated file.
%% 
%% The source is maintained by the LaTeX Project team and bug
%% reports for it can be opened at https://latex-project.org/bugs.html
%% (but please observe conditions on bug reports sent to that address!)
%% 
%% 
%% Copyright (C) 1993-2024
%% The LaTeX Project and any individual authors listed elsewhere
%% in this file.
%% 
%% This file was generated from file(s) of the LaTeX base system.
%% --------------------------------------------------------------
%% 
%% It may be distributed and/or modified under the
%% conditions of the LaTeX Project Public License, either version 1.3c
%% of this license or (at your option) any later version.
%% The latest version of this license is in
%%    https://www.latex-project.org/lppl.txt
%% and version 1.3c or later is part of all distributions of LaTeX
%% version 2008 or later.
%% 
%% This file has the LPPL maintenance status "maintained".
%% 
%% This file may only be distributed together with a copy of the LaTeX
%% base system. You may however distribute the LaTeX base system without
%% such generated files.
%% 
%% The list of all files belonging to the LaTeX base distribution is
%% given in the file `manifest.txt'. See also `legal.txt' for additional
%% information.
%% 
%% The list of derived (unpacked) files belonging to the distribution
%% and covered by LPPL is defined by the unpacking scripts (with
%% extension .ins) which are part of the distribution.
\ifx\newluafunction\undefined\else\expandafter\endinput\fi
\ifx
  \ProvidesFile\undefined\begingroup\def\ProvidesFile
  #1#2[#3]{\endgroup\immediate\write-1{File: #1 #3}}
\fi
\ProvidesFile{ltluatex.tex}%
[2023/08/03 v1.2c
  LuaTeX support for plain TeX (core)
]
\edef\etatcatcode{\the\catcode`\@}
\catcode`\@=11
\ifnum\luatexversion<60 %
  \wlog{***************************************************}
  \wlog{* LuaTeX version too old for ltluatex support *}
  \wlog{***************************************************}
  \expandafter\endinput
\fi
\long\def\@gobble#1{}
\long\def\@firstofone#1{#1}
\directlua{tex.enableprimitives("",tex.extraprimitives("luatex"))}
\ifx\e@alloc\@undefined
  \ifx\documentclass\@undefined
    \ifx\loccount\@undefined
      \input{etex.src}%
    \fi
    \catcode`\@=11 %
    \outer\expandafter\def\csname newfam\endcsname
                          {\alloc@8\fam\chardef\et@xmaxfam}
  \else
    \RequirePackage{etex}
    \expandafter\def\csname newfam\endcsname
                    {\alloc@8\fam\chardef\et@xmaxfam}
    \expandafter\let\expandafter\new@mathgroup\csname newfam\endcsname
  \fi
\edef \et@xmaxregs {\ifx\directlua\@undefined 32768\else 65536\fi}
\edef \et@xmaxfam {\ifx\Umathcode\@undefined\sixt@@n\else\@cclvi\fi}
\count 270=\et@xmaxregs % locally allocates \count registers
\count 271=\et@xmaxregs % ditto for \dimen registers
\count 272=\et@xmaxregs % ditto for \skip registers
\count 273=\et@xmaxregs % ditto for \muskip registers
\count 274=\et@xmaxregs % ditto for \box registers
\count 275=\et@xmaxregs % ditto for \toks registers
\count 276=\et@xmaxregs % ditto for \marks classes
\expandafter\let\csname newcount\expandafter\expandafter\endcsname
                \csname globcount\endcsname
\expandafter\let\csname newdimen\expandafter\expandafter\endcsname
                \csname globdimen\endcsname
\expandafter\let\csname newskip\expandafter\expandafter\endcsname
                \csname globskip\endcsname
\expandafter\let\csname newbox\expandafter\expandafter\endcsname
                \csname globbox\endcsname
\chardef\e@alloc@top=65535
\let\e@alloc@chardef\chardef
\def\e@alloc#1#2#3#4#5#6{%
  \global\advance#3\@ne
  \e@ch@ck{#3}{#4}{#5}#1%
  \allocationnumber#3\relax
  \global#2#6\allocationnumber
  \wlog{\string#6=\string#1\the\allocationnumber}}%
\gdef\e@ch@ck#1#2#3#4{%
  \ifnum#1<#2\else
    \ifnum#1=#2\relax
      #1\@cclvi
      \ifx\count#4\advance#1 10 \fi
    \fi
    \ifnum#1<#3\relax
    \else
      \errmessage{No room for a new \string#4}%
    \fi
  \fi}%
\expandafter\csname newcount\endcsname\e@alloc@attribute@count
\expandafter\csname newcount\endcsname\e@alloc@ccodetable@count
\expandafter\csname newcount\endcsname\e@alloc@luafunction@count
\expandafter\csname newcount\endcsname\e@alloc@whatsit@count
\expandafter\csname newcount\endcsname\e@alloc@bytecode@count
\expandafter\csname newcount\endcsname\e@alloc@luachunk@count
\fi
\ifx\e@alloc@attribute@count\@undefined
  \countdef\e@alloc@attribute@count=258
  \e@alloc@attribute@count=\z@
\fi
\def\newattribute#1{%
  \e@alloc\attribute\attributedef
    \e@alloc@attribute@count\m@ne\e@alloc@top#1%
}
\def\setattribute#1#2{#1=\numexpr#2\relax}
\def\unsetattribute#1{#1=-"7FFFFFFF\relax}
\ifx\e@alloc@ccodetable@count\@undefined
  \countdef\e@alloc@ccodetable@count=259
  \e@alloc@ccodetable@count=\z@
\fi
\def\newcatcodetable#1{%
  \e@alloc\catcodetable\chardef
    \e@alloc@ccodetable@count\m@ne{"8000}#1%
  \initcatcodetable\allocationnumber
}
\newcatcodetable\catcodetable@initex
\newcatcodetable\catcodetable@string
\begingroup
  \def\setrangecatcode#1#2#3{%
    \ifnum#1>#2 %
      \expandafter\@gobble
    \else
      \expandafter\@firstofone
    \fi
      {%
        \catcode#1=#3 %
        \expandafter\setrangecatcode\expandafter
          {\number\numexpr#1 + 1\relax}{#2}{#3}
      }%
  }
  \@firstofone{%
    \catcodetable\catcodetable@initex
      \catcode0=12 %
      \catcode13=12 %
      \catcode37=12 %
      \setrangecatcode{65}{90}{12}%
      \setrangecatcode{97}{122}{12}%
      \catcode92=12 %
      \catcode127=12 %
      \savecatcodetable\catcodetable@string
    \endgroup
  }%
\newcatcodetable\catcodetable@latex
\newcatcodetable\catcodetable@atletter
\begingroup
  \def\parseunicodedataI#1;#2;#3;#4\relax{%
    \parseunicodedataII#1;#3;#2 First>\relax
  }%
  \def\parseunicodedataII#1;#2;#3 First>#4\relax{%
    \ifx\relax#4\relax
      \expandafter\parseunicodedataIII
    \else
      \expandafter\parseunicodedataIV
    \fi
      {#1}#2\relax%
  }%
  \def\parseunicodedataIII#1#2#3\relax{%
    \ifnum 0%
      \if L#21\fi
      \if M#21\fi
      >0 %
      \catcode"#1=11 %
    \fi
  }%
  \def\parseunicodedataIV#1#2#3\relax{%
    \read\unicoderead to \unicodedataline
    \if L#2%
      \count0="#1 %
      \expandafter\parseunicodedataV\unicodedataline\relax
    \fi
  }%
  \def\parseunicodedataV#1;#2\relax{%
    \loop
      \unless\ifnum\count0>"#1 %
        \catcode\count0=11 %
        \advance\count0 by 1 %
    \repeat
  }%
  \def\storedpar{\par}%
  \chardef\unicoderead=\numexpr\count16 + 1\relax
  \openin\unicoderead=UnicodeData.txt %
  \loop\unless\ifeof\unicoderead %
    \read\unicoderead to \unicodedataline
    \unless\ifx\unicodedataline\storedpar
      \expandafter\parseunicodedataI\unicodedataline\relax
    \fi
  \repeat
  \closein\unicoderead
  \@firstofone{%
    \catcode64=12 %
    \savecatcodetable\catcodetable@latex
    \catcode64=11 %
    \savecatcodetable\catcodetable@atletter
   }
\endgroup
\ifx\e@alloc@luafunction@count\@undefined
  \countdef\e@alloc@luafunction@count=260
  \e@alloc@luafunction@count=\z@
\fi
\def\newluafunction{%
  \e@alloc\luafunction\e@alloc@chardef
    \e@alloc@luafunction@count\m@ne\e@alloc@top
}
\def\newluacmd{%
  \e@alloc\luafunction\luadef
    \e@alloc@luafunction@count\m@ne\e@alloc@top
}
\def\newprotectedluacmd{%
  \e@alloc\luafunction{\protected\luadef}
    \e@alloc@luafunction@count\m@ne\e@alloc@top
}
\ifx\e@alloc@whatsit@count\@undefined
  \countdef\e@alloc@whatsit@count=261
  \e@alloc@whatsit@count=\z@
\fi
\def\newwhatsit#1{%
  \e@alloc\whatsit\e@alloc@chardef
    \e@alloc@whatsit@count\m@ne\e@alloc@top#1%
}
\ifx\e@alloc@bytecode@count\@undefined
  \countdef\e@alloc@bytecode@count=262
  \e@alloc@bytecode@count=\z@
\fi
\def\newluabytecode#1{%
  \e@alloc\luabytecode\e@alloc@chardef
    \e@alloc@bytecode@count\m@ne\e@alloc@top#1%
}

\ifx\e@alloc@luachunk@count\@undefined
  \countdef\e@alloc@luachunk@count=263
  \e@alloc@luachunk@count=\z@
\fi
\def\newluachunkname#1{%
  \e@alloc\luachunk\e@alloc@chardef
    \e@alloc@luachunk@count\m@ne\e@alloc@top#1%
    {\escapechar\m@ne
    \directlua{lua.name[\the\allocationnumber]="\string#1"}}%
}
\def\now@and@everyjob#1{%
  \everyjob\expandafter{\the\everyjob
    #1%
  }%
  #1%
}
  \begingroup
    \attributedef\attributezero=0 %
    \chardef     \charzero     =0 %
    \countdef    \CountZero    =0 %
    \dimendef    \dimenzero    =0 %
    \mathchardef \mathcharzero =0 %
    \muskipdef   \muskipzero   =0 %
    \skipdef     \skipzero     =0 %
    \toksdef     \tokszero     =0 %
    \directlua{require("ltluatex")}
  \endgroup
\catcode`\@=\etatcatcode\relax
\endinput
%%
%% End of file `ltluatex.tex'.
