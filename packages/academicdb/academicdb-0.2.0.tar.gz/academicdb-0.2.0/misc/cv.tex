%------------------------------------
% autoCV template - Russell Poldrack, 2022
% License: MIT License
%
% Adapted from original template by: 
% Dario Taraborelli
% Typesetting your academic CV in LaTeX
%
% URL: http://nitens.org/taraborelli/cvtex
% DISCLAIMER: This template is provided for free and without any guarantee 
% that it will correctly compile on your system if you have a non-standard  
% configuration.
%------------------------------------


%!TEX TS-program = xelatex
%!TEX encoding = UTF-8 Unicode

\documentclass[10pt, letterpaper]{article}
\usepackage{fontspec} 

\usepackage{multibib}
\usepackage{multicol}
\usepackage[compact]{titlesec}

\titlespacing{\section}{2pt}{*2}{*0}
\titlespacing{\subsection}{2pt}{*2}{*0}
\titlespacing{\subsubsection}{2pt}{*2}{*0}

% DOCUMENT LAYOUT
\usepackage{geometry} 
\geometry{letterpaper, textwidth=7.25in, textheight=9.5in, marginparsep=7pt, marginparwidth=.6in}
\setlength\parindent{0in}

% FONTS
\defaultfontfeatures{Mapping=tex-text} % converts LaTeX specials (``quotes'' --- dashes etc.) to unicode


% ---- MARGIN YEARS
\newcommand{\years}[1]{\marginpar{\scriptsize #1}}

% HEADINGS
\usepackage{sectsty} 
\usepackage[normalem]{ulem} 
\sectionfont{\rmfamily\mdseries\upshape\Large}
\subsectionfont{\rmfamily\bfseries\upshape\normalsize} 
\subsubsectionfont{\rmfamily\mdseries\upshape\normalsize} 

% PDF SETUP
% ---- FILL IN HERE THE DOC TITLE AND AUTHOR

\usepackage[bookmarks, colorlinks, breaklinks, pdftitle={Curriculum Vitae},pdfauthor={Author}]{hyperref}  
\hypersetup{linkcolor=blue,citecolor=blue,filecolor=black,urlcolor=blue}

\renewcommand\UrlFont{\color{red}\rmfamily\em}

\setmainfont[Ligatures=TeX]{TeX Gyre Termes} 

\begin{document}
\reversemarginpar 
{\LARGE Russell A. Poldrack}\\[4mm] 
\vspace{-1cm} 

\begin{multicols}{2} 
Stanford University\\
Department of Psychology\\
Building 420\\
450 Jane Stanford Way\\
Stanford, CA, 94305-2130\\

\columnbreak 

Phone: 650-497-8488 \\
email: russpold@stanford.edu \\
url: \href{http://www.poldracklab.org}{www.poldracklab.org} \\
url: \href{http://github.com/poldrack}{github.com/poldrack} \\
Twitter: @russpoldrack \\
ORCID: \href{https://orcid.org/0000-0001-6755-0259}{0000-0001-6755-0259} \\
\end{multicols}

\hrule

\section*{Education and training}
\noindent
\textit{1985-1989}: B.A., Baylor University, Waco, TX

\textit{1989-1995}: Ph.D., University of Illinois, Urbana/Champaign, IL

\textit{1995-1999}: Postdoctoral Fellow, Stanford University, Stanford, CA


\section*{Employment and professional affiliations}
\noindent
\textit{2020-present}: Associate Director (Stanford Data Science), Stanford University

\textit{2017-2020}: Professor (by courtesy) (Computer Science), Stanford University

\textit{2016-present}: Albert Ray Lang Professor (Psychology), Stanford University

\textit{2014-present}: Professor (Psychology), Stanford University

\textit{2013-2014}: C. B. Smith, Sr., Nash Phillips, Clyde Copus Centennial Chair (Psychology/Neurobiology), University of Texas

\textit{2009-2014}: Director (Imaging Research Center), University of Texas

\textit{2009-2014}: Professor (Psychology/Neurobiology), University of Texas

\textit{2008-2009}: Professor (Psychology/Psychiatry and Biobehavioral Sciences), UCLA

\textit{2006-2008}: Associate Professor (Psychology/Psychiatry and Biobehavioral Sciences), UCLA

\textit{2002-2006}: Assistant Professor (Psychology), UCLA

\textit{2001-2002}: Member of the Faculty (None), Harvard Graduate School of Education

\textit{1999-2002}: Assistant Psychologist (MGH-NMR Center), Massachusetts General Hospital

\textit{1999-2002}: Assistant Professor (Radiology), Harvard Medical School


\section*{Honors and Awards}
\noindent
\textit{2022}: Open Science Award, Organization for Human Brain Mapping

\textit{2021}: Fellow, Organization for Human Brain Mapping

\textit{2018}: Fellow, Society of Experimental Psychologists

\textit{2018}: Visiting Scholar, IMT Lucca

\textit{2017}: Fellow, Psychonomic Society

\textit{2016}: Distinguished Scholar, Chinese University of Hong Kong

\textit{2012}: Innovation Award for Yarkoni et al., 2011 Nature Methods paper, Social and Affective Neuroscience Society

\textit{2012}: Hilgard Scholar, Stanford University

\textit{2010}: PROSE Awards for Economics and Excellence in the Social Sciences for Neuroeconomics: Decision Making and the Brain, Association of American Publishers

\textit{2010}: Visiting Professor, Beijing Normal University

\textit{2009}: Fellow, Association for Psychological Science

\textit{2005}: Wiley Young Investigator Award, Organization for Human Brain Mapping

\textit{2005}: Distinguished Scientific Award for Early Career Contributions to Psychology, American Psychological Association

\textit{2004}: Brian P. Copenhaver Award for Innovation in Teaching with Technology, UCLA


\section*{Editorial duties}
\noindent
\textit{Founding Co-Editor-in-Chief}: Frontiers in Brain Imaging Methods

\textit{Associate Editor}: Frontiers in Human Neuroscience

\textit{Contributing Editor}: Psychological Bulletin

\textit{Handling Editor (ad hoc) }: Proceedings of the National Academy of Sciences, eLife 

\textit{Editorial board }: Nature Scientific Data (Senior Editorial Board) , Annual Review of Psychology , Trends in Cognitive Sciences , Cerebral Cortex , Human Brain Mapping , GigaScience , NBDT (Neurons, Behavior, Data Analysis, and Theory) 


\section*{Professional societies}
\noindent
Association for Psychological Science, Memory Disorders Research Society, Organization for Human Brain Mapping, Society for Neuroscience


\section*{Service}
\noindent
Member, Board of Scientific Counselors, National Institute of Mental Health, 2022-present

Associate Director, Stanford Data Science Institute, 2020-present

Director, SDS Center for Open and Reproducible Science, 2020-present

Advisory Board Member, XSEDE, 2020-2022

Member (elected), Brain Imaging Data Structure (BIDS) Steering Group, 2019-2022

Scientific Committee Member, ICON (International Conference on Cognitive Neuroscience), 2017-2017

Member, IMH Workgroup on Revisions to the RDoC Matrix`, 2017-2018

Executive Committee Member, NIH Core Neuropsychological Measures for Obesity and Diabetes Project,, 2017-2018

Education Chair (elected), Organization for Human Brain Mapping, 2017-2018

Executive Committee Member, Cognitive Computational Neuroscience Conference, 2016-2022

Member, Editorial Search Committee for \emph{Advances in Methodologies and Practices in Psychological Science.} , 2016-2016

Advisory Panel Member, Databrary, 2016-2020

Co-chair, INCF Congress on Neuroinformatics, 2016-2017

Organizer, Beijing Advanced fMRI Analysis Course, 2015-2015

Member, Behavioral And Social Sciences Workgroup, National Advisory Mental Health Council, 2015-2016

External Advisory Board Member, Adolescent Brain Cognitive Development (ABCD) Study, 2015-2021

External Advisory Board Member, Center for Reproducible Neuroimaging Computation, University of Massachusetts , 2015-present

Member, Advisory Panel to the Scientific Agenda, Kavli Human Project, 2015-2018

Data Intensive Science Advisory Group Member, Texas Advanced Computing Center, 2014-present

Member, OHBM Committee on Best Practice in Data Analysis, 2014-2016

Program Committee Member, Psychonomic Society, 2014-2015

Program Committee Member, INCF Congress on Neuroinformatics, 2013-2014

Standing member, NIH SPC Study Section , 2012-2018

Steering Committee Member, National Academies Keck Futures Initiative on the Informed Brain in a Digital World, 2012-2013

Chair of External Advisory Panel, Human Connectome Project, 2011-2015

Co-Organizer,  INCF Neuroimaging Datasharing and Data Access Workshop, 2011-2011

Selection Committee Member, APA Early Career Award in Behavioral/Cognitive Neuroscience, 2010-2010

Chair (elected), Organization for Human Brain Mapping, 2009-2010

Member, Society for Neuroscience Neuroinformatics Committee, 2008-2010

Co-organizer, UCLA Advanced Neuroimaging Summer School, 2007-2009

Chief Information Officer, Society for Neuroeconomics, 2007-2013

Organizer, OHBM Cognitive Neuroscience Course, 2006-2007

Program Committee Member, Organization for Human Brain Mapping, 2004-2005

Co-organizer, IPAM Summer School on Mathematics in Brain Imaging, 2004-2004

Standing member, National Science Foundation Cognitive Neuroscience Panel, 2003-2005

Executive Committee Member, Cognitive Neuroscience of Category Learning Conference, 2002-2004


\section*{Research funding}
\noindent

\subsection*{Active:}
Principal Investigator, National Institute of Mental Health (\href{https://reporter.nih.gov/project-details/10515980}{\textit{R01MH130898}}), Data-driven validation of cognitive rdoc dimensions using deep phenotyping, 2022-2027\vspace{2mm}

Principal Investigator, Wu Tsai Human Performance Alliance, Precise brain maps of motor skill training in humans, 2022-2023\vspace{2mm}

Principal Investigator, National Institute of Mental Health (\href{https://reporter.nih.gov/project-details/10260312}{\textit{RF1MH121867}}), Nipreps: integrating neuroimaging preprocessing workflows across modalities, populations, and species, 2021-2024\vspace{2mm}

Principal Investigator, National Institute of Mental Health (\href{http://projectreporter.nih.gov/project_info_description.cfm?aid=9770947}{\textit{R24MH117179}}), Openneuro: an open archive for analysis and sharing of brain initiative data, 2018-2023\vspace{2mm}

Principal Investigator, National Institute of Mental Health (\href{http://projectreporter.nih.gov/project_info_description.cfm?aid=9906911}{\textit{R01MH117772}}), Characterizing cognitive control networks using a precision neuroscience approach, 2018-2023\vspace{2mm}

Subcontract PI (F. Pestilli, PI), National Institute of Mental Health (\href{https://reporter.nih.gov/project-details/10253558}{\textit{R01MH126699}}), A community-driven development of the brain imaging data standard (bids) to describe macroscopic brain connections, 2021-2023\vspace{2mm}

Subcontract PI (M. Shenton, PI), National Institute of Mental Health (\href{https://reporter.nih.gov/project-details/10092398}{\textit{U24MH124629}}), Psychosis risk evaluation, data integration and computational technologies (predict): data processing, analysis, and coordination center, 2020-2025\vspace{2mm}

Subcontract PI (S. Makeig, PI), National Institute of Mental Health (\href{http://projectreporter.nih.gov/project_info_description.cfm?aid=9795341}{\textit{R24MH120037}}), Brain initiative resource: development of a human neuroelectromagnetic data archive and tools resource (nemar), 2019-2024\vspace{2mm}

Subcontract PI (T. Yarkoni, PI), National Institute of Mental Health (\href{https://projectreporter.nih.gov/project_info_description.cfm?aid=9881347}{\textit{R01MH096906}}), Large-scale image-based meta-analysis of functional mri data, 2012-2023\vspace{2mm}

\subsection*{Completed:}Subcontract PI (T. Yarkoni, PI), National Institute of Mental Health  (\href{https://projectreporter.nih.gov/project_info_description.cfm?aid=9742532}{\textit{R01MH109682}}), Neuroscout: a cloud-based platform for flexible re-analysis of naturalistic fmri datasets, 2016-2021\vspace{2mm}

Principal Investigator, National Science Foundation  (\href{http://www.nsf.gov/awardsearch/showAward?AWD_ID=1760950&HistoricalAwards=false}{\textit{1760950}}), Breaking down barriers for reproducible neuroimaging data analyses, 2018-2021\vspace{2mm}

Principal Investigator, National Institute on Drug Abuse  (\href{http://projectreporter.nih.gov/project_info_description.cfm?aid=9780476}{\textit{UH3DA041713}}), Applying novel technologies and methods to inform the ontology of self-regulation, 2018-2021\vspace{2mm}

Co-I (L. Williams, PI), National Institute of Mental Health  (\href{https://projectreporter.nih.gov/project_info_description.cfm?aid=9925811}{\textit{U01MH109985}}), Mapping connectomes for disordered emotional states, 2017-2021\vspace{2mm}

Subcontract Pi (R Adolphs, PI), National Institute of Neurological Disorders and Stroke  (\href{http://projectreporter.nih.gov/project_info_description.cfm?aid=9830084}{\textit{U01NS103780}}), Causal mapping of emotion networks with concurrent electrical stimulation and fmri, 2017-2020\vspace{2mm}

Principal Investigator, National Institute of Mental Health  (\href{http://projectreporter.nih.gov/project_info_description.cfm?aid=9566304}{\textit{R24MH114705}}), Bids-derivatives: a data standard for derived data and models in the brain initiative, 2017-2020\vspace{2mm}

Principal Investigator, National Science Foundation  (\href{http://www.nsf.gov/awardsearch/showAward?AWD_ID=1649658&HistoricalAwards=false}{\textit{1649658}}), Computational infrastructure for brain research: eager: a computationally enabled knowledge infrastructure for cognitive neuroscience, 2017-2020\vspace{2mm}

Subcontract PI (S. Ghosh, PI), National Institute of Biomedical Imaging and Bioengineering  (\href{https://projectreporter.nih.gov/project_info_description.cfm?aid=9603723}{\textit{R01EB020740}}), Nipype: dataflows for reproducible biomedical research, 2016-2020\vspace{2mm}

Subcontract PI (D.Glahn, PI), National Institute of Mental Health  (\href{http://projectreporter.nih.gov/project_info_description.cfm?aid=9634110}{\textit{R01MH106324}}), Gene networks influencing psychotic dysconnectivity in african americans, 2014-2019\vspace{2mm}

Subcontract PI (L. Marsch, PI), National Institute on Drug Abuse  (\href{http://projectreporter.nih.gov/project_info_description.cfm?aid=9310428}{\textit{UH2DA041713}}), Applying novel technologies and methods to inform the ontology of self-regulation, 2015-2018\vspace{2mm}

Principal Investigator, Laura and John Arnold Foundation , Stanford center for reproducible neuroscience, 2015-2018\vspace{2mm}

Collaborator (G. Leng, PI), European Community FP7 , The neurobiology of decision-making in eating-innovative tools (nudge- it), 2014-2018\vspace{2mm}

Co-I (J. Davis, PI), National Institute of Diabetes and Digestive and Kidney Diseases  (\href{http://projectreporter.nih.gov/project_info_description.cfm?aid=8829825}{\textit{R21DK098719}}), Sugar sweetened beverages: impact on reward, satiety, and metabolism in children, 2014-2017\vspace{2mm}

Principal Investigator, National Institute on Drug Abuse  (\href{http://projectreporter.nih.gov/project_info_description.cfm?aid=8662735}{\textit{R21DA034316}}), The development of neural responses to punishment in adolescence, 2013-2017\vspace{2mm}

Principal Investigator, National Institute on Aging  (\href{http://projectreporter.nih.gov/project_info_description.cfm?aid=8968152}{\textit{R01AG041653}}), Overcoming the persistence of first-learned habits to maintain behavioral change, 2011-2017\vspace{2mm}

Co-PI, Stanford Cyber Initiative , Behavioral metrics for cyber authentication , 2015-2017\vspace{2mm}

Principal Investigator, National Science Foundation  (\href{http://grants.uberresearch.com/100000001/1131441/CRCNS-Data-Sharing-An-open-data-repository-for-cognitive-neuroscience-The-OpenfMRI-Project}{\textit{1131441}}), An open data repository for cognitive neuroscience: the openfmri project, 2011-2014\vspace{2mm}

Principal Investigator, National Institute of Mental Health  (\href{http://projectreporter.nih.gov/project_info_description.cfm?aid=8228112}{\textit{R01MH082795}}), The cognitive atlas: developing an interdisciplinary knowledge base through socia, 2008-2014\vspace{2mm}

PI, Office of Naval Research , Acquisition of an mri-compatible eeg system , 2013-2014\vspace{2mm}

PI, Office of Naval Research , Predicting individual differences using resting-state fmri and network analysis , 2010-2014\vspace{2mm}

Principal Investigator, National Center for Advancing Translational Sciences  (\href{http://projectreporter.nih.gov/project_info_description.cfm?aid=7934944}{\textit{G20RR030871}}), Enhancing an imaging core at the university of texas at austin, 2010-2013\vspace{2mm}

Co-I, NIH Roadmap  (\href{https://projectreporter.nih.gov/project_info_description.cfm?aid=7650594}{\textit{PL1MH083271}}), Consortium for neuropsychiatric phenomics , 2007-2012\vspace{2mm}

Co-I (S. Hanson, PI), James S. McDonnell Foundation  (\href{https://www.jsmf.org/grants/2009003/}{\textit{2009003}}), Accessing brain interactivity ii, 2008-2011\vspace{2mm}

Co-I, National Institute of Mental Health  (\href{https://projectreporter.nih.gov/project_info_description.cfm?aid=7480935}{\textit{P50MH077248}}), Cidar: translational research to enhance cognitive control (trecc), 2006-2011\vspace{2mm}

Principal Investigator, James S. McDonnell Foundation  (\href{https://www.jsmf.org/grants/2005013/}{\textit{2005013}}), Habit, automaticity, and cognitive control, 2004-2009\vspace{2mm}

Co-PI, National Science Foundation  (\href{http://grants.uberresearch.com/100000001/0433693/The-Neural-Basis-of-Risky-Decision-Making}{\textit{433693}}), The neural basis of risky decision making, 2004-2008\vspace{2mm}

Principal Investigator, National Science Foundation  (\href{http://grants.uberresearch.com/100000001/0223843/COLLABORATIVE-RESEARCH-The-Cognitive-Neuroscience-of-Category-Learning}{\textit{223843}}), The cognitive neuroscience of category learning, 2003-2007\vspace{2mm}

Principal Investigator, National Institute of Neurological Disorders and Stroke  (\href{http://projectreporter.nih.gov/project_info_description.cfm?aid=6623403}{\textit{R21NS043333}}), Cholinergic enhancement of human cortical plasticity, 2002-2004\vspace{2mm}

PI, Janssen Pharmaceuticals , Cholinergic enhancement of perceptual learning , 2002-2003\vspace{2mm}

Principal Investigator, National Science Foundation  (\href{http://grants.uberresearch.com/100000001/0121950/Enhancing-Human-Cortical-Plasticity-Visual-Psychophysics-and-fMRI}{\textit{121950}}), Enhancing human cortical plasticity: visual psychophysics and fmri, 2001-2002\vspace{2mm}

PI, Alafi Family Foundation , Multimodal imaging of reading development and dyslexia , 2000-2002\vspace{2mm}

Fellow, McDonnell-Pew Program for Cognitive Neuroscience , The neural basis of skill learning using fmri , 1996-1999\vspace{2mm}

Fellow, National Institute of Mental Health  (\href{http://projectreporter.nih.gov/project_info_description.cfm?aid=2241597}{\textit{F31MH010433}}), Relational representation in memory and amnesia, 1994-1994\vspace{2mm}


\section*{Teaching}
\noindent
\textit{Undergraduate}: Introduction to Statistics, Judgment and Decision Making, Reading the Brain (Intro Seminar), Introduction to Cognitive Science, Cognitive Neuroscience of Memory, Functional MRI Laboratory\vspace{2mm}

\textit{Graduate}: Cognitive Neuroscience, Brain Networks, Advanced Statistical Modeling, Functional Neuroimaging, Neuroeconomics, Human Learning and Memory, Computer Methods for Experimental Psychology\vspace{2mm}


\section*{Publications}
\noindent
\subsection*{2022}Bissett PG, Poldrack RA.  (2022). Estimating the Time to Do Nothing: Toward Next-Generation Models of Response Inhibition. \textit{Current Directions in Psychological Science}. \href{https://doi.org/10.1177/09637214221121753}{DOI} \vspace{2mm}

Ciric R,  Thomas AW, Esteban O,  Poldrack RA.  (2022). Differentiable programming for functional connectomics. \textit{Proceedings of the 2nd Machine Learning for Health symposium, PMLR , 193.0}, 419-455. \vspace{2mm}

Ciric R, Thompson WH, Lorenz R et al. (2022). TemplateFlow: FAIR-sharing of multi-scale, multi-species brain models. \textit{Nature Methods, 19}, 1568-1571. \href{https://www.ncbi.nlm.nih.gov/pmc/articles/PMC9718663}{OA} \href{https://doi.org/10.1038/s41592-022-01681-2}{DOI} \vspace{2mm}

Vega A, Rocca R, Blair RW, Markiewicz CJ, Mentch J, Kent JD, Herholz P, Ghosh SS, Poldrack RA, Yarkoni T.  (2022). Neuroscout, a unified platform for generalizable and reproducible fMRI research. \textit{eLife, 11}, e79277. \href{https://www.ncbi.nlm.nih.gov/pmc/articles/PMC9489206}{OA} \href{https://doi.org/10.7554/elife.79277}{DOI} \vspace{2mm}

Jwa AS, Poldrack RA.  (2022). The spectrum of data sharing policies in neuroimaging data repositories. \textit{Human Brain Mapping, 43}, 2707-2721. \href{https://www.ncbi.nlm.nih.gov/pmc/articles/PMC9057092}{OA} \href{https://doi.org/10.1002/hbm.25803}{DOI} \vspace{2mm}

Paret C, Unverhau N, Feingold F, Poldrack RA, Stirner M, Schmahl C, Sicorello M.  (2022). Survey on Open Science Practices in Functional Neuroimaging. \textit{NeuroImage, 257}, 119306. \href{https://doi.org/10.1016/j.neuroimage.2022.119306}{DOI} \vspace{2mm}

Poline JB, Kennedy DN, Sommer FT et al. (2022). Is Neuroscience FAIR? A Call for Collaborative Standardisation of Neuroscience Data. \textit{Neuroinformatics, 20}, 507-512. \href{https://www.ncbi.nlm.nih.gov/pmc/articles/PMC9300762}{OA} \href{https://doi.org/10.1007/s12021-021-09557-0}{DOI} \vspace{2mm}

Scherer EA, Metcalf SA, Whicker CL et al. (2022). Momentary Influences on Self-Regulation in Two Populations With Health Risk Behaviors: Adults Who Smoke and Adults Who Are Overweight and Have Binge-Eating Disorder. \textit{Frontiers in Digital Health, 4}, 798895. \href{https://doi.org/10.3389/fdgth.2022.798895}{DOI} \vspace{2mm}

Scholz C, Chan HY, Poldrack RA, Ridder DT, Smidts A, Laan LN.  (2022). Can we have a second helping? A preregistered direct replication study on the neurobiological mechanisms underlying self-control. \textit{Human Brain Mapping, 43}, 4995-5016. \href{https://www.ncbi.nlm.nih.gov/pmc/articles/PMC9582371}{OA} \href{https://doi.org/10.1002/hbm.26065}{DOI} \vspace{2mm}

Thomas AW, Ré C, Poldrack RA.  (2022). Self-Supervised Learning of Brain Dynamics from Broad Neuroimaging Data. \textit{Advances in Neural Information Processing Systems 35 (NeurIPS 2022)}. \vspace{2mm}

Thomas AW, Ré C, Poldrack RA.  (2022). Interpreting mental state decoding with deep learning models. \textit{Trends in Cognitive Sciences, 26}, 972-986. \href{https://doi.org/10.1016/j.tics.2022.07.003}{DOI} \vspace{2mm}

Vaghi MM, Hagen MK, Jones HM, Mumford JA, Bissett PG, Poldrack RA.  (2022). Relating psychiatric symptoms and self-regulation during the COVID-19 crisis. \textit{Translational Psychiatry, 12}, 271. \href{https://www.ncbi.nlm.nih.gov/pmc/articles/PMC9274960}{OA} \href{https://doi.org/10.1038/s41398-022-02030-9}{DOI} \vspace{2mm}

Walters J, King M, Bissett PG, Ivry RB, Diedrichsen J, Poldrack RA.  (2022). Predicting brain activation maps for arbitrary tasks with cognitive encoding models. \textit{NeuroImage, 263}, 119610. \href{https://doi.org/10.1016/j.neuroimage.2022.119610}{DOI} \vspace{2mm}

Wimmer GE, Poldrack RA.  (2022). Reward learning and working memory: Effects of massed versus spaced training and post-learning delay period. \textit{Memory and Cognition, 50}, 312-324. \href{https://www.ncbi.nlm.nih.gov/pmc/articles/PMC8821056}{OA} \href{https://doi.org/10.3758/s13421-021-01233-7}{DOI} \vspace{2mm}

\subsection*{2021}Aczel B, Szaszi B, Nilsonne G et al. (2021). Consensus-based guidance for conducting and reporting multi-analyst studies. \textit{eLife, 10}, e72185. \href{https://www.ncbi.nlm.nih.gov/pmc/articles/PMC8626083}{OA} \href{https://doi.org/10.7554/elife.72185}{DOI} \vspace{2mm}

Beam E, Potts C, Poldrack RA, Etkin A.  (2021). A data-driven framework for mapping domains of human neurobiology. \textit{Nature Neuroscience, 24}, 1733-1744. \href{https://www.ncbi.nlm.nih.gov/pmc/articles/PMC8761068}{OA} \href{https://doi.org/10.1038/s41593-021-00948-9}{DOI} \vspace{2mm}

Bissett PG, Jones HM, Poldrack RA, Logan GD.  (2021). Severe violations of independence in response inhibition tasks. \textit{Science Advances, 7}, eabf4355. \href{https://www.ncbi.nlm.nih.gov/pmc/articles/PMC7968836}{OA} \href{https://doi.org/10.1126/sciadv.abf4355}{DOI} \vspace{2mm}

Bissett PG, Hagen MP, Jones HM, Poldrack RA.  (2021). Design issues and solutions for stop-signal data from the adolescent brain cognitive development (Abcd) study. \textit{eLife, 10}, e60185. \href{https://www.ncbi.nlm.nih.gov/pmc/articles/PMC7997655}{OA} \href{https://doi.org/10.7554/elife.60185}{DOI} \vspace{2mm}

Enkavi AZ, Poldrack RA.  (2021). Implications of the Lacking Relationship Between Cognitive Task and Self-report Measures for Psychiatry. \textit{Biological Psychiatry: Cognitive Neuroscience and Neuroimaging, 6}, 670-672. \href{https://doi.org/10.1016/j.bpsc.2020.06.010}{DOI} \vspace{2mm}

Fox AS, Holley D, Klink PC et al. (2021). Sharing voxelwise neuroimaging results from rhesus monkeys and other species with Neurovault. \textit{NeuroImage, 225}, 117518. \href{https://www.ncbi.nlm.nih.gov/pmc/articles/PMC7846271}{OA} \href{https://doi.org/10.1016/j.neuroimage.2020.117518}{DOI} \vspace{2mm}

Lam M, Chen CY, Ge T et al. (2021). Identifying nootropic drug targets via large-scale cognitive GWAS and transcriptomics. \textit{Neuropsychopharmacology, 46}, 1788-1801. \href{https://www.ncbi.nlm.nih.gov/pmc/articles/PMC8357785}{OA} \href{https://doi.org/10.1038/s41386-021-01023-4}{DOI} \vspace{2mm}

Levitis E, Praag CD, Gau R et al. (2021). Centering inclusivity in the design of online conferences - An OHBM-Open Science perspective. \textit{GigaScience, 10}, giab051. \href{https://www.ncbi.nlm.nih.gov/pmc/articles/PMC8377301}{OA} \href{https://doi.org/10.1093/gigascience/giab051}{DOI} \vspace{2mm}

Manapat PD, Edwards MC, MacKinnon DP, Poldrack RA, Marsch LA.  (2021). A Psychometric Analysis of the Brief Self-Control Scale. \textit{Assessment, 28}, 395-412. \href{https://www.ncbi.nlm.nih.gov/pmc/articles/PMC7261631}{OA} \href{https://doi.org/10.1177/1073191119890021}{DOI} \vspace{2mm}

Markiewicz CJ, Gorgolewski KJ, Feingold F et al. (2021). The openneuro resource for sharing of neuroscience data. \textit{eLife, 10}, e71774. \href{https://www.ncbi.nlm.nih.gov/pmc/articles/PMC8550750}{OA} \href{https://doi.org/10.7554/elife.71774}{DOI} \vspace{2mm}

Mazza GL, Smyth HL, Bissett PG et al. (2021). Correlation Database of 60 Cross-Disciplinary Surveys and Cognitive Tasks Assessing Self-Regulation. \textit{Journal of Personality Assessment, 103}, 238-245. \href{https://www.ncbi.nlm.nih.gov/pmc/articles/PMC7483539}{OA} \href{https://doi.org/10.1080/00223891.2020.1732994}{DOI} \href{https://github.com/IanEisenberg/Self_Regulation_Ontology/tree/master/Data}{Data} \vspace{2mm}

Poldrack RA.  (2021).  \textit{Hard To Break: Why Our Brains Make Habits Stick}. Princeton, NJ: Princeton University Press.\vspace{2mm}

Poldrack RA.  (2021). The physics of representation. \textit{Synthese, 199}, 1307-1325. \href{https://doi.org/10.1007/s11229-020-02793-y}{OA} \href{https://doi.org/10.1007/s11229-020-02793-y}{DOI} \vspace{2mm}

Poldrack RA.  (2021). Diving into the deep end: a personal reflection on the MyConnectome study. \textit{Current Opinion in Behavioral Sciences, 40}, 1-4. \href{https://doi.org/10.1016/j.cobeha.2020.10.008}{DOI} \vspace{2mm}

Zmigrod L, Eisenberg IW, Bissett PG, Robbins TW, Poldrack RA.  (2021). The cognitive and perceptual correlates of ideological attitudes: A data-driven approach. \textit{Philosophical Transactions of the Royal Society B: Biological Sciences, 376}, 20200424. \href{https://www.ncbi.nlm.nih.gov/pmc/articles/PMC7935109}{OA} \href{https://doi.org/10.1098/rstb.2020.0424}{DOI} \vspace{2mm}

\subsection*{2020}Aron AR, Ivry RB, Jeffery KJ, Poldrack RA, Schmidt R, Summerfield C, Urai AE.  (2020). How Can Neuroscientists Respond to the Climate Emergency?. \textit{Neuron, 106}, 17-20. \href{https://doi.org/10.1016/j.neuron.2020.02.019}{DOI} \vspace{2mm}

Bassett DS, Cullen KE, Eickhoff SB et al. (2020). Reflections on the past two decades of neuroscience. \textit{Nature Reviews Neuroscience, 21}, 524-534. \href{https://doi.org/10.1038/s41583-020-0363-6}{DOI} \vspace{2mm}

Bielczyk NZ, Ando A, Badhwar AP et al. (2020). Effective Self-Management for Early Career Researchers in the Natural and Life Sciences. \textit{Neuron, 106}, 212-217. \href{https://www.ncbi.nlm.nih.gov/pmc/articles/PMC7665085}{OA} \href{https://doi.org/10.1016/j.neuron.2020.03.015}{DOI} \href{https://osf.io/w6emk/}{OSF} \vspace{2mm}

Botvinik-Nezer R, Holzmeister F, Camerer CF et al. (2020). Variability in the analysis of a single neuroimaging dataset by many teams. \textit{Nature, 582}, 84-88. \href{https://www.ncbi.nlm.nih.gov/pmc/articles/PMC7771346}{OA} \href{https://doi.org/10.1038/s41586-020-2314-9}{DOI} \vspace{2mm}

Colenbier N, Steen F, Uddin LQ, Poldrack RA, Calhoun VD, Marinazzo D.  (2020). Disambiguating the role of blood flow and global signal with partial information decomposition. \textit{NeuroImage, 213}, 116699. \href{https://doi.org/10.1016/j.neuroimage.2020.116699}{DOI} \vspace{2mm}

Dockès J, Poldrack RA, Primet R, Gözükan H, Yarkoni T, Suchanek F, Thirion B, Varoquaux G.  (2020). Neuroquery, comprehensive meta-analysis of human brain mapping. \textit{eLife, 9}. \href{https://www.ncbi.nlm.nih.gov/pmc/articles/PMC7164961}{OA} \href{https://doi.org/10.7554/elife.53385}{DOI} \vspace{2mm}

Esteban O, Goncalves M, Markiewicz CJ, Ghosh SS, Poldrack RA.  (2020). Software Tool to Read, Represent, Manipulate, and Apply N-Dimensional Spatial Transforms. \textit{Proceedings - International Symposium on Biomedical Imaging, 2020-April}, 709-712. \href{https://doi.org/10.1109/isbi45749.2020.9098466}{DOI} \vspace{2mm}

Esteban O, Ciric R, Finc K et al. (2020). Analysis of task-based functional MRI data preprocessed with fMRIPrep. \textit{Nature Protocols, 15}, 2186-2202. \href{https://www.ncbi.nlm.nih.gov/pmc/articles/PMC7404612}{OA} \href{https://doi.org/10.1038/s41596-020-0327-3}{DOI} \vspace{2mm}

Lurie DJ, Kessler D, Bassett DS et al. (2020). Questions and controversies in the study of time-varying functional connectivity in resting fMRI. \textit{Network Neuroscience, 4}, 30-69. \href{https://doi.org/10.1162/netn_a_00116}{DOI} \href{https://osf.io/fa6zr/}{OSF} \vspace{2mm}

Mollon J, Mathias SR, Knowles EE et al. (2020). Cognitive impairment from early to middle adulthood in patients with affective and nonaffective psychotic disorders. \textit{Psychological Medicine, 50}, 48-57. \href{https://www.ncbi.nlm.nih.gov/pmc/articles/PMC7086288}{OA} \href{https://doi.org/10.1017/s0033291718003938}{DOI} \vspace{2mm}

Poldrack RA, Whitaker K, Kennedy D.  (2020). Introduction to the special issue on reproducibility in neuroimaging. \textit{NeuroImage, 218}, 116357. \href{https://doi.org/10.1016/j.neuroimage.2019.116357}{DOI} \vspace{2mm}

Poldrack RA, Huckins G, Varoquaux G.  (2020). Establishment of Best Practices for Evidence for Prediction: A Review. \textit{JAMA Psychiatry, 77}, 534-540. \href{https://www.ncbi.nlm.nih.gov/pmc/articles/PMC7250718}{OA} \href{https://doi.org/10.1001/jamapsychiatry.2019.3671}{DOI} \vspace{2mm}

Thompson WH, Nair R, Oya H, Esteban O, Shine JM, Petkov CI, Poldrack RA, Howard M, Adolphs R.  (2020). A data resource from concurrent intracranial stimulation and functional MRI of the human brain. \textit{Scientific Data, 7}, 258. \href{https://www.ncbi.nlm.nih.gov/pmc/articles/PMC7406507}{OA} \href{https://doi.org/10.1038/s41597-020-00595-y}{DOI} \vspace{2mm}

Thompson WH, Kastrati G, Finc K, Wright J, Shine JM, Poldrack RA.  (2020). Time-varying nodal measures with temporal community structure: A cautionary note to avoid misinterpretation. \textit{Human Brain Mapping, 41}, 2347-2356. \href{https://www.ncbi.nlm.nih.gov/pmc/articles/PMC7268033}{OA} \href{https://doi.org/10.1002/hbm.24950}{DOI} \vspace{2mm}

Thompson WH, Wright J, Bissett PG, Poldrack RA.  (2020). Dataset decay and the problem of sequential analyses on open datasets. \textit{eLife, 9}, 1-17. \href{https://www.ncbi.nlm.nih.gov/pmc/articles/PMC7237204}{OA} \href{https://doi.org/10.7554/elife.53498}{DOI} \vspace{2mm}

Tozzi L, Staveland B, Holt-Gosselin B et al. (2020). The human connectome project for disordered emotional states: Protocol and rationale for a research domain criteria study of brain connectivity in young adult anxiety and depression. \textit{NeuroImage, 214}, 116715. \href{https://www.ncbi.nlm.nih.gov/pmc/articles/PMC8597395}{OA} \href{https://doi.org/10.1016/j.neuroimage.2020.116715}{DOI} \vspace{2mm}

\subsection*{2019}Aridan N, Malecek NJ, Poldrack RA, Schonberg T.  (2019). Neural correlates of effort-based valuation with prospective choices. \textit{NeuroImage, 185}, 446-454. \href{https://www.ncbi.nlm.nih.gov/pmc/articles/PMC6289638}{OA} \href{https://doi.org/10.1016/j.neuroimage.2018.10.051}{DOI} \href{https://openneuro.org/datasets/ds001167/versions/00002}{Data} \vspace{2mm}

Botvinik-Nezer R, Iwanir R, Holzmeister F, Huber J, Johannesson M, Kirchler M, Dreber A, Camerer CF, Poldrack RA, Schonberg T.  (2019). fMRI data of mixed gambles from the Neuroimaging Analysis Replication and Prediction Study. \textit{Scientific Data, 6}, 106. \href{https://www.ncbi.nlm.nih.gov/pmc/articles/PMC6602933}{OA} \href{https://doi.org/10.1038/s41597-019-0113-7}{DOI} \href{https://openneuro.org/datasets/ds001734/versions/1.0.5}{Data} \vspace{2mm}

Eisenberg IW, Bissett PG, Enkavi A, Li J, MacKinnon DP, Marsch LA, Poldrack RA.  (2019). Uncovering the structure of self-regulation through data-driven ontology discovery. \textit{Nature Communications, 10}, 2319. \href{https://www.ncbi.nlm.nih.gov/pmc/articles/PMC6534563}{OA} \href{https://doi.org/10.1038/s41467-019-10301-1}{DOI} \href{https://github.com/IanEisenberg/Self_Regulation_Ontology/tree/master/Data}{Data} \href{https://github.com/IanEisenberg/Self_Regulation_Ontology}{Code} \href{https://osf.io/zk6w9/}{OSF} \vspace{2mm}

Esteban O, Blair RW, Nielson DM, Varada JC, Marrett S, Thomas AG, Poldrack RA, Gorgolewski KJ.  (2019). Crowdsourced MRI quality metrics and expert quality annotations for training of humans and machines. \textit{Scientific Data, 6}, 30. \href{https://www.ncbi.nlm.nih.gov/pmc/articles/PMC6472378}{OA} \href{https://doi.org/10.1038/s41597-019-0035-4}{DOI} \vspace{2mm}

Esteban O, Markiewicz CJ, Blair RW et al. (2019). fMRIPrep: a robust preprocessing pipeline for functional MRI. \textit{Nature Methods, 16}, 111-116. \href{https://www.ncbi.nlm.nih.gov/pmc/articles/PMC6319393}{OA} \href{https://doi.org/10.1038/s41592-018-0235-4}{DOI} \vspace{2mm}

Kebets V, Holmes AJ, Orban C, Tang S, Li J, Sun N, Kong R, Poldrack RA, Yeo BT.  (2019). Somatosensory-Motor Dysconnectivity Spans Multiple Transdiagnostic Dimensions of Psychopathology. \textit{Biological Psychiatry, 86}, 779-791. \href{https://doi.org/10.1016/j.biopsych.2019.06.013}{DOI} \vspace{2mm}

King M, Hernandez-Castillo CR, Poldrack RA, Ivry RB, Diedrichsen J.  (2019). Functional boundaries in the human cerebellum revealed by a multi-domain task battery. \textit{Nature Neuroscience, 22}, 1371-1378. \href{https://www.ncbi.nlm.nih.gov/pmc/articles/PMC8312478}{OA} \href{https://doi.org/10.1038/s41593-019-0436-x}{DOI} \href{https://openneuro.org/datasets/ds002105/versions/1.1.0}{Data} \vspace{2mm}

Lam M, Hill WD, Trampush JW et al. (2019). Pleiotropic Meta-Analysis of Cognition, Education, and Schizophrenia Differentiates Roles of Early Neurodevelopmental and Adult Synaptic Pathways. \textit{American Journal of Human Genetics, 105}, 334-350. \href{https://www.ncbi.nlm.nih.gov/pmc/articles/PMC6699140}{OA} \href{https://doi.org/10.1016/j.ajhg.2019.06.012}{DOI} \vspace{2mm}

Li M, Han Y, Aburn MJ, Breakspear M, Poldrack RA, Shine JM, Lizier JT.  (2019). Transitions in information processing dynamics at the whole-brain network level are driven by alterations in neural gain. \textit{PLoS computational biology, 15}, e1006957. \href{https://www.ncbi.nlm.nih.gov/pmc/articles/PMC6793849}{OA} \href{https://doi.org/10.1371/journal.pcbi.1006957}{DOI} \vspace{2mm}

Poldrack RA, Gorgolewski KJ, Varoquaux G.  (2019). Computational and Informatic Advances for Reproducible Data Analysis in Neuroimaging. \textit{Annual Review of Biomedical Data Science, 2.0}, 119-138. \href{https://doi.org/10.1146/annurev-biodatasci-072018-021237}{DOI} \vspace{2mm}

Poldrack RA, Feingold F, Frank MJ et al. (2019). The Importance of Standards for Sharing of Computational Models and Data. \textit{Computational Brain and Behavior, 2}, 229-232. \href{https://doi.org/10.1007/s42113-019-00062-x}{DOI} \vspace{2mm}

Poldrack RA.  (2019). The Costs of Reproducibility. \textit{Neuron, 101}, 11-14. \href{https://doi.org/10.1016/j.neuron.2018.11.030}{OA} \href{https://doi.org/10.1016/j.neuron.2018.11.030}{DOI} \vspace{2mm}

Reid AT, Headley DB, Mill RD et al. (2019). Advancing functional connectivity research from association to causation. \textit{Nature Neuroscience, 22}, 1751-1760. \href{https://www.ncbi.nlm.nih.gov/pmc/articles/PMC7289187}{OA} \href{https://doi.org/10.1038/s41593-019-0510-4}{DOI} \vspace{2mm}

Shine JM, Hearne LJ, Breakspear M, Hwang K, Müller EJ, Sporns O, Poldrack RA, Mattingley JB, Cocchi L.  (2019). The Low-Dimensional Neural Architecture of Cognitive Complexity Is Related to Activity in Medial Thalamic Nuclei. \textit{Neuron, 104}, 849-855.e3. \href{https://doi.org/10.1016/j.neuron.2019.09.002}{OA} \href{https://doi.org/10.1016/j.neuron.2019.09.002}{DOI} \vspace{2mm}

Shine JM, Breakspear M, Bell PT, Martens KA, Shine R, Koyejo O, Sporns O, Poldrack RA.  (2019). Publisher Correction: Human cognition involves the dynamic integration of neural activity and neuromodulatory systems (Nature Neuroscience, (2019), 22, 2, (289-296), 10.1038/s41593-018-0312-0). \textit{Nature Neuroscience, 22}, 1036. \href{https://doi.org/10.1038/s41593-019-0347-x}{OA} \href{https://doi.org/10.1038/s41593-019-0347-x}{DOI} \vspace{2mm}

Shine JM, Bell PT, Matar E, Poldrack RA, Lewis SJ, Halliday GM, O'Callaghan C.  (2019). Dopamine depletion alters macroscopic network dynamics in Parkinson's disease. \textit{Brain, 142}, 1024-1034. \href{https://www.ncbi.nlm.nih.gov/pmc/articles/PMC6904322}{OA} \href{https://doi.org/10.1093/brain/awz034}{DOI} \vspace{2mm}

Shine JM, Breakspear M, Bell PT, Martens K, Shine R, Koyejo O, Sporns O, Poldrack RA.  (2019). Human cognition involves the dynamic integration of neural activity and neuromodulatory systems. \textit{Nature Neuroscience, 22}, 289-296. \href{https://doi.org/10.1038/s41593-018-0312-0}{DOI} \vspace{2mm}

Smeets PA, Dagher A, Hare TA, Kullmann S, Laan LN, Poldrack RA, Preissl H, Small D, Stice E, Veldhuizen MG.  (2019). Good practice in food-related neuroimaging. \textit{American Journal of Clinical Nutrition, 109}, 491-503. \href{https://www.ncbi.nlm.nih.gov/pmc/articles/PMC7945961}{OA} \href{https://doi.org/10.1093/ajcn/nqy344}{DOI} \vspace{2mm}

Varoquaux G, Poldrack RA.  (2019). Predictive models avoid excessive reductionism in cognitive neuroimaging. \textit{Current Opinion in Neurobiology, 55}, 1-6. \href{https://doi.org/10.1016/j.conb.2018.11.002}{DOI} \vspace{2mm}

Verbruggen F, Aron AR, Band GP et al. (2019). A consensus guide to capturing the ability to inhibit actions and impulsive behaviors in the stop-signal task. \textit{eLife, 8}, e46323. \href{https://www.ncbi.nlm.nih.gov/pmc/articles/PMC6533084}{OA} \href{https://doi.org/10.7554/elife.46323}{DOI} \vspace{2mm}

Enkavi A, Eisenberg IW, Bissett PG, Mazza GL, MacKinnon DP, Marsch LA, Poldrack RA.  (2019). Right measures for the research question. \textit{Proceedings of the National Academy of Sciences of the United States of America, 116}, 24398-24399. \href{https://www.ncbi.nlm.nih.gov/pmc/articles/PMC6900533}{OA} \href{https://doi.org/10.1073/pnas.1917123116}{DOI} \vspace{2mm}

Enkavi A, Eisenberg IW, Bissett PG, Mazza GL, MacKinnon DP, Marsch LA, Poldrack RA.  (2019). Large-scale analysis of test–retest reliabilities of self-regulation measures. \textit{Proceedings of the National Academy of Sciences of the United States of America, 116}, 5472-5477. \href{https://www.ncbi.nlm.nih.gov/pmc/articles/PMC6431228}{OA} \href{https://doi.org/10.1073/pnas.1818430116}{DOI} \href{https://github.com/IanEisenberg/Self_Regulation_Ontology/tree/master/Data}{Data} \href{https://osf.io/5mjns/}{OSF} \vspace{2mm}

Zuo XN, Biswal BB, Poldrack RA.  (2019). Editorial: Reliability and reproducibility in functional connectomics. \textit{Frontiers in Neuroscience, 13}, 117. \href{https://doi.org/10.3389/fnins.2019.00117}{DOI} \vspace{2mm}

\subsection*{2018}Andersen MR, Winther O, Hansen LK, Poldrack R, Koyejo O.  (2018). Bayesian structure learning for dynamic brain connectivity. \textit{International Conference on Artificial Intelligence and Statistics, AISTATS 2018}, 1436-1446. \vspace{2mm}

Bakkour A, Botvinik-Nezer R, Cohen N, Hover AM, Poldrack RA, Schonberg T.  (2018). Spacing of cue-approach training leads to better maintenance of behavioral change. \textit{PLoS ONE, 13}, e0201580. \href{https://www.ncbi.nlm.nih.gov/pmc/articles/PMC6066248}{OA} \href{https://doi.org/10.1371/journal.pone.0201580}{DOI} \href{https://osf.io/fdvrk/}{OSF} \vspace{2mm}

Davies G, Lam M, Harris SE et al. (2018). Study of 300,486 individuals identifies 148 independent genetic loci influencing general cognitive function. \textit{Nature Communications, 9}, 2098. \href{https://www.ncbi.nlm.nih.gov/pmc/articles/PMC5974083}{OA} \href{https://doi.org/10.1038/s41467-018-04362-x}{DOI} \vspace{2mm}

Dockès J, Wassermann D, Poldrack R, Suchanek F, Thirion B, Varoquaux G.  (2018). Text to brain: predicting the spatial distribution of neuroimaging observations from text reports. \textit{Lecture Notes in Computer Science (including subseries Lecture Notes in Artificial Intelligence and Lecture Notes in Bioinformatics), 11072 LNCS}, 584-592. \href{https://doi.org/10.1007/978-3-030-00931-1_67}{DOI} \vspace{2mm}

Eisenberg IW, Bissett PG, Canning JR et al. (2018). Applying novel technologies and methods to inform the ontology of self-regulation. \textit{Behaviour Research and Therapy, 101}, 46-57. \href{https://www.ncbi.nlm.nih.gov/pmc/articles/PMC5801197}{OA} \href{https://doi.org/10.1016/j.brat.2017.09.014}{DOI} \href{https://osf.io/amxpv/}{OSF} \vspace{2mm}

Esteban O, Poldrack RA, Gorgolewski KJ.  (2018). Improving out-of-sample prediction of quality of MRIQC. \textit{Lecture Notes in Computer Science (including subseries Lecture Notes in Artificial Intelligence and Lecture Notes in Bioinformatics), 11043 LNCS}, 190-199. \href{https://doi.org/10.1007/978-3-030-01364-6_21}{DOI} \vspace{2mm}

Gorgolewski KJ, Nichols T, Kennedy DN, Poline JB, Poldrack RA.  (2018). Making replication prestigious. \textit{The Behavioral and brain sciences, 41}, e131. \href{https://doi.org/10.1017/s0140525x18000663}{DOI} \vspace{2mm}

Lam M, Trampush JW, Yu J et al. (2018). Multi-Trait analysis of gwas and biological insights into cognition: A response to hill (2018). \textit{Twin Research and Human Genetics, 21}, 394-397. \href{https://doi.org/10.1017/thg.2018.46}{DOI} \href{https://osf.io/t5kzx/}{OSF} \vspace{2mm}

Mathias SR, Knowles EE, Barrett J, Beetham T, Leach O, Buccheri S, Aberizk K, Blangero J, Poldrack RA, Glahn DC.  (2018). Deficits in visual working-memory capacity and general cognition in African Americans with psychosis. \textit{Schizophrenia Research, 193}, 100-106. \href{https://www.ncbi.nlm.nih.gov/pmc/articles/PMC5825248}{OA} \href{https://doi.org/10.1016/j.schres.2017.08.015}{DOI} \vspace{2mm}

Naselaris T, Bassett DS, Fletcher AK, Kording K, Kriegeskorte N, Nienborg H, Poldrack RA, Shohamy D, Kay K.  (2018). Cognitive Computational Neuroscience: A New Conference for an Emerging Discipline. \textit{Trends in Cognitive Sciences, 22}, 365-367. \href{https://www.ncbi.nlm.nih.gov/pmc/articles/PMC5911192}{OA} \href{https://doi.org/10.1016/j.tics.2018.02.008}{DOI} \vspace{2mm}

Poldrack RA.  (2018).  \textit{The New Mind Readers: What Neuroimaging Can and Cannot Reveal about our Thoughts}. Princeton, NJ: Princeton University Press.\vspace{2mm}

Poldrack RA.  (2018).  \textit{Statistical Thinking for the 21st Century}. http://statsthinking21.org.\vspace{2mm}

Poldrack RA, Sandak R.  (2018).  \textit{The cognitive neuroscience of reading: A special issue of scientific studies of reading}. .\href{https://doi.org/10.4324/9780203764442}{DOI} \vspace{2mm}

Poldrack RA, Monahan J, Imrey PB, Reyna V, Raichle ME, Faigman D, Buckholtz JW.  (2018). Predicting Violent Behavior: What Can Neuroscience Add?. \textit{Trends in Cognitive Sciences, 22}, 111-123. \href{https://www.ncbi.nlm.nih.gov/pmc/articles/PMC5794654}{OA} \href{https://doi.org/10.1016/j.tics.2017.11.003}{DOI} \href{https://osf.io/tgknp/}{OSF} \vspace{2mm}

Savage JE, Jansen PR, Stringer S et al. (2018). Genome-wide association meta-analysis in 269,867 individuals identifies new genetic and functional links to intelligence. \textit{Nature Genetics, 50}, 912-919. \href{https://www.ncbi.nlm.nih.gov/pmc/articles/PMC6411041}{OA} \href{https://doi.org/10.1038/s41588-018-0152-6}{DOI} \vspace{2mm}

Shine JM, Poldrack RA.  (2018). Principles of dynamic network reconfiguration across diverse brain states. \textit{NeuroImage, 180}, 396-405. \href{https://doi.org/10.1016/j.neuroimage.2017.08.010}{DOI} \vspace{2mm}

Shine JM, Aburn MJ, Breakspear M, Poldrack RA.  (2018). The modulation of neural gain facilitates a transition between functional segregation and integration in the brain. \textit{eLife, 7}, e31130. \href{https://www.ncbi.nlm.nih.gov/pmc/articles/PMC5818252}{OA} \href{https://doi.org/10.7554/elife.31130}{DOI} \vspace{2mm}

Shine JM, Brink RL, Hernaus D, Nieuwenhuis S, Poldrack RA.  (2018). Catecholaminergic manipulation alters dynamic network topology across cognitive states. \textit{Network Neuroscience, 2}, 381-396. \href{https://doi.org/10.1162/netn_a_00042}{DOI} \vspace{2mm}

Tansey W, Koyejo O, Poldrack RA, Scott JG.  (2018). False Discovery Rate Smoothing. \textit{Journal of the American Statistical Association, 113}, 1156-1171. \href{https://doi.org/10.1080/01621459.2017.1319838}{DOI} \vspace{2mm}

Varoquaux G, Schwartz Y, Poldrack RA, Gauthier B, Bzdok D, Poline JB, Thirion B.  (2018). Atlases of cognition with large-scale human brain mapping. \textit{PLoS Computational Biology, 14}, e1006565. \href{https://www.ncbi.nlm.nih.gov/pmc/articles/PMC6289578}{OA} \href{https://doi.org/10.1371/journal.pcbi.1006565}{DOI} \vspace{2mm}

White C, Poldrack RA.  (2018). Methods for fMRI analysis. In \textit{The Stevens’ Handbook of Experimental Psychology and Cognitive Neuroscience, Fourth Edition (Volume 5)}. \href{https://doi.org/10.1002/9781119170174.epcn515}{DOI} \vspace{2mm}

Wimmer GE, Li JK, Gorgolewski KJ, Poldrack RA.  (2018). Reward learning over weeks versus minutes increases the neural representation of value in the human brain. \textit{Journal of Neuroscience, 38}, 7649-7666. \href{https://www.ncbi.nlm.nih.gov/pmc/articles/PMC6113901}{OA} \href{https://doi.org/10.1523/jneurosci.0075-18.2018}{DOI} \href{https://osf.io/z2gwf/}{OSF} \vspace{2mm}

\subsection*{2017}Acikalin MY, Gorgolewski KJ, Poldrack RA.  (2017). A coordinate-based meta-analysis of overlaps in regional specialization and functional connectivity across subjective value and default mode networks. \textit{Frontiers in Neuroscience, 11}, 1. \href{https://doi.org/10.3389/fnins.2017.00001}{DOI} \vspace{2mm}

Bakkour A, Lewis-Peacock JA, Poldrack RA, Schonberg T.  (2017). Neural mechanisms of cue-approach training. \textit{NeuroImage, 151}, 92-104. \href{https://www.ncbi.nlm.nih.gov/pmc/articles/PMC5365383}{OA} \href{https://doi.org/10.1016/j.neuroimage.2016.09.059}{DOI} \vspace{2mm}

Eckert MA, Vaden KI, Maxwell AB et al. (2017). Common Brain Structure Findings Across Children with Varied Reading Disability Profiles. \textit{Scientific Reports, 7}, 6009. \href{https://www.ncbi.nlm.nih.gov/pmc/articles/PMC5519686}{OA} \href{https://doi.org/10.1038/s41598-017-05691-5}{DOI} \vspace{2mm}

Esteban O, Birman D, Schaer M, Koyejo OO, Poldrack RA, Gorgolewski KJ.  (2017). MRIQC: Advancing the automatic prediction of image quality in MRI from unseen sites. \textit{PLoS ONE, 12}, e0184661. \href{https://www.ncbi.nlm.nih.gov/pmc/articles/PMC5612458}{OA} \href{https://doi.org/10.1371/journal.pone.0184661}{DOI} \href{https://osf.io/haf97/}{OSF} \vspace{2mm}

Gilron R, Rosenblatt J, Koyejo O, Poldrack RA, Mukamel R.  (2017). What's in a pattern? Examining the type of signal multivariate analysis uncovers at the group level. \textit{NeuroImage, 146}, 113-120. \href{https://doi.org/10.1016/j.neuroimage.2016.11.019}{DOI} \vspace{2mm}

Gorgolewski KJ, Alfaro-Almagro F, Auer T et al. (2017). BIDS apps: Improving ease of use, accessibility, and reproducibility of neuroimaging data analysis methods. \textit{PLoS Computational Biology, 13}, e1005209. \href{https://www.ncbi.nlm.nih.gov/pmc/articles/PMC5363996}{OA} \href{https://doi.org/10.1371/journal.pcbi.1005209}{DOI} \vspace{2mm}

Gorgolewski KJ, Durnez J, Poldrack RA.  (2017). Preprocessed Consortium for Neuropsychiatric Phenomics dataset. \textit{F1000Research, 6}, 1262. \href{https://doi.org/10.12688/f1000research.11964.2}{DOI} \vspace{2mm}

Hodgson K, Poldrack RA, Curran JE et al. (2017). Shared Genetic Factors Influence Head Motion during MRI and Body Mass Index. \textit{Cerebral Cortex, 27}, 5539-5546. \href{https://www.ncbi.nlm.nih.gov/pmc/articles/PMC6075600}{OA} \href{https://doi.org/10.1093/cercor/bhw321}{DOI} \vspace{2mm}

Khanna R, Ghosh J, Poldrack RA, Koyejo O.  (2017). A Deflation Method for Structured Probabilistic PCA. \textit{Proceedings of the 2017 SIAM International Conference on Data Mining}. \href{https://doi.org/10.1137/1.9781611974973.60}{DOI} \vspace{2mm}

Khanna R, Ghosh J, Poldrack R, Koyejo O.  (2017). Information projection and approximate inference for structured sparse variables. \textit{Proceedings of the 20th International Conference on Artificial Intelligence and Statistics, AISTATS 2017}. \vspace{2mm}

Khanna R, Ghosh J, Poldrack R, Koyejo O.  (2017). A deflation method for structured probabilistic PCA. \textit{Proceedings of the 17th SIAM International Conference on Data Mining, SDM 2017}, 534-542. \href{https://doi.org/10.1137/1.9781611974973.60}{OA} \href{https://doi.org/10.1137/1.9781611974973.60}{DOI} \vspace{2mm}

Kiar G, Gorgolewski KJ, Kleissas D et al. (2017). Science in the cloud (SIC): A use case in MRI connectomics. \textit{GigaScience, 6}, 1-10. \href{https://www.ncbi.nlm.nih.gov/pmc/articles/PMC5467033}{OA} \href{https://doi.org/10.1093/gigascience/gix013}{DOI} \vspace{2mm}

Lam M, Trampush JW, Yu J et al. (2017). Large-Scale Cognitive GWAS Meta-Analysis Reveals Tissue-Specific Neural Expression and Potential Nootropic Drug Targets. \textit{Cell Reports, 21}, 2597-2613. \href{https://www.ncbi.nlm.nih.gov/pmc/articles/PMC5789458}{OA} \href{https://doi.org/10.1016/j.celrep.2017.11.028}{DOI} \vspace{2mm}

Mathias SR, Knowles EE, Barrett J, Leach O, Buccheri S, Beetham T, Blangero J, Poldrack RA, Glahn DC.  (2017). The Processing-Speed Impairment in Psychosis Is More Than Just Accelerated Aging. \textit{Schizophrenia Bulletin, 43}, 814-823. \href{https://www.ncbi.nlm.nih.gov/pmc/articles/PMC5472152}{OA} \href{https://doi.org/10.1093/schbul/sbw168}{DOI} \vspace{2mm}

Nichols TE, Das S, Eickhoff SB et al. (2017). Best practices in data analysis and sharing in neuroimaging using MRI. \textit{Nature Neuroscience, 20}, 299-303. \href{https://www.ncbi.nlm.nih.gov/pmc/articles/PMC5685169}{OA} \href{https://doi.org/10.1038/nn.4500}{DOI} \vspace{2mm}

Poldrack R.  (2017). Developing a reproducible workflow for large-scale phenotyping. In \textit{The Practice of Reproducible Research: Case Studies and Lessons from the Data-Intensive Sciences}, 311-316. \vspace{2mm}

Poldrack R.  (2017). Neuroscience: The risks of reading the brain. \textit{Nature, 541}, 156. \href{https://doi.org/10.1038/541156a}{OA} \href{https://doi.org/10.1038/541156a}{DOI} \vspace{2mm}

Poldrack RA.  (2017). Precision Neuroscience: Dense Sampling of Individual Brains. \textit{Neuron, 95}, 727-729. \href{https://doi.org/10.1016/j.neuron.2017.08.002}{OA} \href{https://doi.org/10.1016/j.neuron.2017.08.002}{DOI} \vspace{2mm}

Poldrack RA, Baker CI, Durnez J, Gorgolewski KJ, Matthews PM, Munafò MR, Nichols TE, Poline JB, Vul E, Yarkoni T.  (2017). Scanning the horizon: Towards transparent and reproducible neuroimaging research. \textit{Nature Reviews Neuroscience, 18}, 115-126. \href{https://www.ncbi.nlm.nih.gov/pmc/articles/PMC6910649}{OA} \href{https://doi.org/10.1038/nrn.2016.167}{DOI} \href{https://osf.io/spr9a/}{OSF} \vspace{2mm}

Poldrack RA, Gorgolewski KJ.  (2017). OpenfMRI: Open sharing of task fMRI data. \textit{NeuroImage, 144}, 259-261. \href{https://www.ncbi.nlm.nih.gov/pmc/articles/PMC4669234}{OA} \href{https://doi.org/10.1016/j.neuroimage.2015.05.073}{DOI} \vspace{2mm}

Rubin TN, Koyejo O, Gorgolewski KJ, Jones MN, Poldrack RA, Yarkoni T.  (2017). Decoding brain activity using a large-scale probabilistic functional-anatomical atlas of human cognition. \textit{PLoS Computational Biology, 13}, e1005649. \href{https://www.ncbi.nlm.nih.gov/pmc/articles/PMC5683652}{OA} \href{https://doi.org/10.1371/journal.pcbi.1005649}{DOI} \vspace{2mm}

Shine JM, Kucyi A, Foster BL, Bickel S, Wang D, Liu H, Poldrack RA, Hsieh LT, Hsiang JC, Parvizi J.  (2017). Distinct patterns of temporal and directional connectivity among intrinsic networks in the human brain. \textit{Journal of Neuroscience, 37}, 9667-9674. \href{https://www.ncbi.nlm.nih.gov/pmc/articles/PMC6596608}{OA} \href{https://doi.org/10.1523/jneurosci.1574-17.2017}{DOI} \vspace{2mm}

Trampush JW, Yang ML, Yu J et al. (2017). GWAS meta-analysis reveals novel loci and genetic correlates for general cognitive function: A report from the COGENT consortium. \textit{Molecular Psychiatry, 22}, 336-345. \href{https://www.ncbi.nlm.nih.gov/pmc/articles/PMC5322272}{OA} \href{https://doi.org/10.1038/mp.2016.244}{DOI} \vspace{2mm}

Xiao X, Dong Q, Gao J, Men W, Poldrack RA, Xue G.  (2017). Transformed neural pattern reinstatement during episodic memory retrieval. \textit{Journal of Neuroscience, 37}, 2986-2998. \href{https://www.ncbi.nlm.nih.gov/pmc/articles/PMC6596730}{OA} \href{https://doi.org/10.1523/jneurosci.2324-16.2017}{DOI} \vspace{2mm}

\subsection*{2016}Asteris M, Kyrillidis A, Koyejo O, Poldrack R.  (2016). A simple and provable algorithm for sparse diagonal CCA. \textit{33rd International Conference on Machine Learning, ICML 2016, 3}, 1774-1783. \vspace{2mm}

Bakkour A, Leuker C, Hover AM, Giles N, Poldrack RA, Schonberg T.  (2016). Mechanisms of choice behavior shift using cue-approach training. \textit{Frontiers in Psychology, 7}, 421. \href{https://doi.org/10.3389/fpsyg.2016.00421}{DOI} \vspace{2mm}

Eckert MA, Berninger VW, Hoeft F et al. (2016). A case of Bilateral Perisylvian Syndrome with reading disability. \textit{Cortex, 76}, 121-124. \href{https://www.ncbi.nlm.nih.gov/pmc/articles/PMC4776332}{OA} \href{https://doi.org/10.1016/j.cortex.2016.01.004}{DOI} \vspace{2mm}

Eisenberg I, Poldrack RA.  (2016). Task-set Selection in Probabilistic Environments: a Model of Task-set Inference. \textit{Proceedings of Cognitive Science Society}. \vspace{2mm}

Gorgolewski KJ, Poldrack RA.  (2016). A Practical Guide for Improving Transparency and Reproducibility in Neuroimaging Research. \textit{PLoS Biology, 14}, e1002506. \href{https://www.ncbi.nlm.nih.gov/pmc/articles/PMC4936733}{OA} \href{https://doi.org/10.1371/journal.pbio.1002506}{DOI} \vspace{2mm}

Gorgolewski KJ, Auer T, Calhoun VD et al. (2016). The brain imaging data structure, a format for organizing and describing outputs of neuroimaging experiments. \textit{Scientific Data, 3}, 160044. \href{https://www.ncbi.nlm.nih.gov/pmc/articles/PMC4978148}{OA} \href{https://doi.org/10.1038/sdata.2016.44}{DOI} \vspace{2mm}

Gorgolewski KJ, Varoquaux G, Rivera G et al. (2016). NeuroVault.org: A repository for sharing unthresholded statistical maps, parcellations, and atlases of the human brain. \textit{NeuroImage, 124}, 1242-1244. \href{https://www.ncbi.nlm.nih.gov/pmc/articles/PMC4806527}{OA} \href{https://doi.org/10.1016/j.neuroimage.2015.04.016}{DOI} \vspace{2mm}

Hastings J, Frishkoff GA, Smith B, Jensen M, Poldrack RA, Lomax J, Bandrowski A, Imam F, Turner JA, Martone ME.  (2016). Interdyscyplinarne perspektywy rozwoju, integracji i zastosowań ontologii poznawczych. \textit{Avant, 7}, 101-117. \href{https://doi.org/10.26913/70302016.0109.0007}{DOI} \vspace{2mm}

Lenartowicz A, Poldrack RA.  (2016). Brain imaging. In \textit{The Curated Reference Collection in Neuroscience and Biobehavioral Psychology}, 187-193. \href{https://doi.org/10.1016/b978-0-12-809324-5.00274-1}{DOI} \vspace{2mm}

Patterson TK, Lenartowicz A, Berkman ET, Ji D, Poldrack RA, Knowlton BJ.  (2016). Putting the brakes on the brakes: negative emotion disrupts cognitive control network functioning and alters subsequent stopping ability. \textit{Experimental Brain Research, 234}, 3107-3118. \href{https://www.ncbi.nlm.nih.gov/pmc/articles/PMC5073018}{OA} \href{https://doi.org/10.1007/s00221-016-4709-2}{DOI} \vspace{2mm}

Poldrack RA, Congdon E, Triplett W et al. (2016). A phenome-wide examination of neural and cognitive function. \textit{Scientific Data, 3}, 160110. \href{https://www.ncbi.nlm.nih.gov/pmc/articles/PMC5139672}{OA} \href{https://doi.org/10.1038/sdata.2016.110}{DOI} \href{https://openneuro.org/datasets/ds000030/versions/1.0.0}{Data} \vspace{2mm}

Poldrack RA, Yarkoni T.  (2016). From brain maps to cognitive ontologies: Informatics and the search for mental structure. \textit{Annual Review of Psychology, 67}, 587-612. \href{https://www.ncbi.nlm.nih.gov/pmc/articles/PMC4701616}{OA} \href{https://doi.org/10.1146/annurev-psych-122414-033729}{DOI} \vspace{2mm}

Poldrack RA, Kittur A, Kalar D, Miller E, Seppa C, Gil Y, Parker D, Sabb FW, Bilder RM.  (2016). Atlas poznawczy: W strone fundamentów wiedzy w neurokognitywistyce. \textit{Avant, 7}, 75-100. \href{https://doi.org/10.26913/70302016.0109.0006}{DOI} \vspace{2mm}

Shine JM, Bissett PG, Bell PT, Koyejo O, Balsters JH, Gorgolewski KJ, Moodie CA, Poldrack RA.  (2016). The Dynamics of Functional Brain Networks: Integrated Network States during Cognitive Task Performance. \textit{Neuron, 92}, 544-554. \href{https://www.ncbi.nlm.nih.gov/pmc/articles/PMC5073034}{OA} \href{https://doi.org/10.1016/j.neuron.2016.09.018}{DOI} \vspace{2mm}

Shine JM, Koyejo O, Poldrack RA.  (2016). Temporal metastates are associated with differential patterns of time-resolved connectivity, network topology, and attention. \textit{Proceedings of the National Academy of Sciences of the United States of America, 113}, 9888-9891. \href{https://www.ncbi.nlm.nih.gov/pmc/articles/PMC5024627}{OA} \href{https://doi.org/10.1073/pnas.1604898113}{DOI} \vspace{2mm}

Shine JM, Eisenberg I, Poldrack RA.  (2016). Computational specificity in the human brain. \textit{Behavioral and Brain Sciences, 39}. \href{https://doi.org/10.1017/s0140525x1500165x}{DOI} \vspace{2mm}

Sochat VV, Eisenberg IW, Enkavi AZ, Li J, Bissett PG, Poldrack RA.  (2016). The experiment factory: Standardizing behavioral experiments. \textit{Frontiers in Psychology, 7}, 610. \href{https://doi.org/10.3389/fpsyg.2016.00610}{DOI} \vspace{2mm}

Wager TD, Atlas LY, Botvinick MM, Chang LJ, Coghill RC, Davis KD, Iannetti GD, Poldrack RA, Shackman AJ, Yarkoni T.  (2016). Pain in the ACC?. \textit{Proceedings of the National Academy of Sciences of the United States of America, 113}, E2474-E2475. \href{https://www.ncbi.nlm.nih.gov/pmc/articles/PMC4983860}{OA} \href{https://doi.org/10.1073/pnas.1600282113}{DOI} \vspace{2mm}

Wiener M, Sommer FT, Ives ZG, Poldrack RA, Litt B.  (2016). Enabling an Open Data Ecosystem for the Neurosciences. \textit{Neuron, 92}, 617-621. \href{https://doi.org/10.1016/j.neuron.2016.10.037}{OA} \href{https://doi.org/10.1016/j.neuron.2016.10.037}{DOI} \vspace{2mm}

Worthy DA, Davis T, Gorlick MA, Cooper JA, Bakkour A, Mumford JA, Poldrack RA, Maddox W.  (2016). Neural correlates of state-based decision-making in younger and older adults. \textit{NeuroImage, 130}, 13-23. \href{https://www.ncbi.nlm.nih.gov/pmc/articles/PMC4808466}{OA} \href{https://doi.org/10.1016/j.neuroimage.2015.12.004}{DOI} \vspace{2mm}

\subsection*{2015}Chen MY, Jimura K, White CN, Maddox W, Poldrack RA.  (2015). Multiple brain networks contribute to the acquisition of bias in perceptual decision-making. \textit{Frontiers in Neuroscience, 9}, 63. \href{https://doi.org/10.3389/fnins.2015.00063}{DOI} \vspace{2mm}

Gorgolewski KJ, Varoquaux G, Rivera G et al. (2015). NeuroVault.Org: A web-based repository for collecting and sharing unthresholded statistical maps of the human brain. \textit{Frontiers in Neuroinformatics, 9}, 8. \href{https://doi.org/10.3389/fninf.2015.00008}{DOI} \vspace{2mm}

Helfinstein SM, Mumford JA, Poldrack RA.  (2015). If all your friends jumped off a bridge: The effect of others' actions on engagement in and recommendation of risky behaviors. \textit{Journal of Experimental Psychology: General, 144}, 12-17. \href{https://doi.org/10.1037/xge0000043}{DOI} \vspace{2mm}

Khanna R, Ghosh J, Poldrack RA, Koyejo O.  (2015). Sparse submodular probabilistic PCA. \textit{Proceedings of the 18th International conference on Artificial Intelligence and Statistics (AISTATS) }. \vspace{2mm}

Laumann TO, Gordon EM, Adeyemo B et al. (2015). Functional System and Areal Organization of a Highly Sampled Individual Human Brain. \textit{Neuron, 87}, 657-670. \href{https://www.ncbi.nlm.nih.gov/pmc/articles/PMC4642864}{OA} \href{https://doi.org/10.1016/j.neuron.2015.06.037}{DOI} \href{https://openneuro.org/datasets/ds000031/versions/00001}{Data} \vspace{2mm}

Mumford JA, Poline JB, Poldrack RA.  (2015). Orthogonalization of regressors in fMRI models. \textit{PLoS ONE, 10}, e0126255. \href{https://www.ncbi.nlm.nih.gov/pmc/articles/PMC4412813}{OA} \href{https://doi.org/10.1371/journal.pone.0126255}{DOI} \vspace{2mm}

Poldrack RA.  (2015). Reverse Inference. In \textit{Brain Mapping: An Encyclopedic Reference (Vol. 1)}, 647-650. \href{https://doi.org/10.1016/b978-0-12-397025-1.00346-8}{DOI} \vspace{2mm}

Poldrack R.  (2015). Introduction to cognitive neuroscience. In \textit{Brain Mapping: An Encyclopedic Reference (Vol. 3)}, 259-260. \href{https://doi.org/10.1016/b978-0-12-397025-1.09990-5}{DOI} \vspace{2mm}

Poldrack RA, Laumann TO, Koyejo O et al. (2015). Long-term neural and physiological phenotyping of a single human. \textit{Nature Communications, 6}, 8885. \href{https://www.ncbi.nlm.nih.gov/pmc/articles/PMC4682164}{OA} \href{https://doi.org/10.1038/ncomms9885}{DOI} \href{https://openneuro.org/datasets/ds000031/versions/00001}{Data} \href{https://github.com/poldrack/myconnectome}{Code} \vspace{2mm}

Poldrack RA, Farah MJ.  (2015). Progress and challenges in probing the human brain. \textit{Nature, 526}, 371-379. \href{https://doi.org/10.1038/nature15692}{DOI} \vspace{2mm}

Poldrack RA, Poline JB.  (2015). The publication and reproducibility challenges of shared data. \textit{Trends in Cognitive Sciences, 19}, 59-61. \href{https://doi.org/10.1016/j.tics.2014.11.008}{DOI} \vspace{2mm}

Poldrack RA.  (2015). Is "efficiency" a useful concept in cognitive neuroscience?. \textit{Developmental Cognitive Neuroscience, 11}, 12-17. \href{https://www.ncbi.nlm.nih.gov/pmc/articles/PMC6989750}{OA} \href{https://doi.org/10.1016/j.dcn.2014.06.001}{DOI} \href{https://github.com/poldrack/rtmodel}{Code} \vspace{2mm}

Shine JM, Koyejo O, Bell PT, Gorgolewski KJ, Gilat M, Poldrack RA.  (2015). Estimation of dynamic functional connectivity using Multiplication of Temporal Derivatives. \textit{NeuroImage, 122}, 399-407. \href{https://doi.org/10.1016/j.neuroimage.2015.07.064}{DOI} \vspace{2mm}

Sochat VV, Gorgolewski KJ, Koyejo O, Durnez J, Poldrack RA.  (2015). Effects of thresholding on correlation-based image similarity metrics. \textit{Frontiers in Neuroscience, 9}, 418. \href{https://doi.org/10.3389/fnins.2015.00418}{DOI} \vspace{2mm}

\subsection*{2014}Aron AR, Robbins TW, Poldrack RA.  (2014). Right inferior frontal cortex: Addressing the rebuttals. \textit{Frontiers in Human Neuroscience, 8}, 905. \href{https://doi.org/10.3389/fnhum.2014.00905}{DOI} \vspace{2mm}

Aron AR, Robbins TW, Poldrack RA.  (2014). Inhibition and the right inferior frontal cortex: One decade on. \textit{Trends in Cognitive Sciences, 18}, 177-185. \href{https://doi.org/10.1016/j.tics.2013.12.003}{DOI} \vspace{2mm}

Congdon E, Altshuler LL, Mumford JA et al. (2014). Neural activation during response inhibition in adult attention-deficit/hyperactivity disorder: Preliminary findings on the effects of medication and symptom severity. \textit{Psychiatry Research - Neuroimaging, 222}, 17-28. \href{https://www.ncbi.nlm.nih.gov/pmc/articles/PMC4009011}{OA} \href{https://doi.org/10.1016/j.pscychresns.2014.02.002}{DOI} \vspace{2mm}

Davis T, LaRocque KF, Mumford JA, Norman KA, Wagner AD, Poldrack RA.  (2014). What do differences between multi-voxel and univariate analysis mean? How subject-, voxel-, and trial-level variance impact fMRI analysis. \textit{NeuroImage, 97}, 271-283. \href{https://www.ncbi.nlm.nih.gov/pmc/articles/PMC4115449}{OA} \href{https://doi.org/10.1016/j.neuroimage.2014.04.037}{DOI} \vspace{2mm}

Davis T, Poldrack RA.  (2014). Quantifying the internal structure of categories using a neural typicality measure. \textit{Cerebral Cortex, 24}, 1720-1737. \href{https://doi.org/10.1093/cercor/bht014}{OA} \href{https://doi.org/10.1093/cercor/bht014}{DOI} \vspace{2mm}

Davis T, Xue G, Love BC, Preston AR, Poldrack RA.  (2014). Global neural pattern similarity as a common basis for categorization and recognition memory. \textit{Journal of Neuroscience, 34}, 7472-7484. \href{https://www.ncbi.nlm.nih.gov/pmc/articles/PMC4035513}{OA} \href{https://doi.org/10.1523/jneurosci.3376-13.2014}{DOI} \vspace{2mm}

Hastings J, Frishkoff GA, Smith B, Jensen M, Poldrack RA, Lomax J, Bandrowski A, Imam F, Turner JA, Martone ME.  (2014). Interdisciplinary perspectives on the development, integration, and application of cognitive ontologies. \textit{Frontiers in Neuroinformatics, 8}, 62. \href{https://doi.org/10.3389/fninf.2014.00062}{DOI} \vspace{2mm}

Helfinstein SM, Schonberg T, Congdon E, Karlsgodt KH, Mumford JA, Sabb FW, Cannon TD, London ED, Bilder RM, Poldrack RA.  (2014). Predicting risky choices from brain activity patterns. \textit{Proceedings of the National Academy of Sciences of the United States of America, 111}, 2470-2475. \href{https://www.ncbi.nlm.nih.gov/pmc/articles/PMC3932884}{OA} \href{https://doi.org/10.1073/pnas.1321728111}{DOI} \href{https://openneuro.org/datasets/ds000030/versions/1.0.0}{Data} \vspace{2mm}

Jimura K, Cazalis F, Stover ER, Poldrack RA.  (2014). The neural basis of task switching changes with skill acquisition. \textit{Frontiers in Human Neuroscience, 8}, 339. \href{https://doi.org/10.3389/fnhum.2014.00339}{DOI} \vspace{2mm}

Koyejo O, Khanna R, Ghosh J, Poldrack RA.  (2014). On Prior Distributions and Approximate Inference for Structured Variables. \textit{Advances in Neural Information Processing Systems, 27.0}. \vspace{2mm}

Koyejo O, Khanna R, Ghosh J, Poldrack RA.  (2014). On prior distributions and approximate inference for structured variables. \textit{Advances in Neural Information Processing Systems, 1}, 676-684. \vspace{2mm}

Mumford JA, Davis T, Poldrack RA.  (2014). The impact of study design on pattern estimation for single-trial multivariate pattern analysis. \textit{NeuroImage, 103}, 130-138. \href{https://doi.org/10.1016/j.neuroimage.2014.09.026}{DOI} \vspace{2mm}

Poldrack RA, Gorgolewski KJ.  (2014). Making big data open: Data sharing in neuroimaging. \textit{Nature Neuroscience, 17}, 1510-1517. \href{https://doi.org/10.1038/nn.3818}{DOI} \vspace{2mm}

Schonberg T, Bakkour A, Hover AM, Mumford JA, Poldrack RA.  (2014). Influencing food choices by training: Evidence for modulation of frontoparietal control signals. \textit{Journal of Cognitive Neuroscience, 26}, 247-268. \href{https://www.ncbi.nlm.nih.gov/pmc/articles/PMC4066661}{OA} \href{https://doi.org/10.1162/jocn_a_00495}{DOI} \vspace{2mm}

Schonberg T, Bakkour A, Hover AM, Mumford JA, Nagar L, Perez J, Poldrack RA.  (2014). Changing value through cued approach: An automatic mechanism of behavior change. \textit{Nature Neuroscience, 17}, 625-630. \href{https://www.ncbi.nlm.nih.gov/pmc/articles/PMC4041518}{OA} \href{https://doi.org/10.1038/nn.3673}{DOI} \vspace{2mm}

Thakkar KN, Congdon E, Poldrack RA, Sabb FW, London ED, Cannon TD, Bilder RM.  (2014). Women are more sensitive than men to prior trial events on the Stop-signal task. \textit{British Journal of Psychology, 105}, 254-272. \href{https://www.ncbi.nlm.nih.gov/pmc/articles/PMC4000536}{OA} \href{https://doi.org/10.1111/bjop.12034}{DOI} \vspace{2mm}

Wagshal D, Knowlton BJ, Cohen JR, Poldrack RA, Bookheimer SY, Bilder RM, Asarnow RF.  (2014). Impaired automatization of a cognitive skill in first-degree relatives of patients with schizophrenia. \textit{Psychiatry Research, 215}, 294-299. \href{https://www.ncbi.nlm.nih.gov/pmc/articles/PMC4191851}{OA} \href{https://doi.org/10.1016/j.psychres.2013.11.024}{DOI} \vspace{2mm}

Wagshal D, Knowlton BJ, Suthana NA, Cohen JR, Poldrack RA, Bookheimer SY, Bilder RM, Asarnow RF.  (2014). Evidence for corticostriatal dysfunction during cognitive skill learning in adolescent siblings of patients with childhood-onset schizophrenia. \textit{Schizophrenia Bulletin, 40}, 1030-1039. \href{https://www.ncbi.nlm.nih.gov/pmc/articles/PMC4133665}{OA} \href{https://doi.org/10.1093/schbul/sbt147}{DOI} \vspace{2mm}

White CN, Poldrack RA.  (2014). Decomposing bias in different types of simple decisions. \textit{Journal of Experimental Psychology: Learning Memory and Cognition, 40}, 385-398. \href{https://doi.org/10.1037/a0034851}{DOI} \vspace{2mm}

White CN, Congdon E, Mumford JA, Karlsgodt KH, Sabb FW, Freimer NB, London ED, Cannon TD, Bilder RM, Poldrack RA.  (2014). Decomposing Decision Components in the Stop-signal Task: A Model-based Approach to Individual Differences in Inhibitory Control. \textit{Journal of Cognitive Neuroscience, 26}, 1601-1614. \href{https://www.ncbi.nlm.nih.gov/pmc/articles/PMC4119005}{OA} \href{https://doi.org/10.1162/jocn_a_00567}{DOI} \href{https://openneuro.org/datasets/ds000030/versions/1.0.0}{Data} \vspace{2mm}

\subsection*{2013}Barch DM, Burgess GC, Harms MP et al. (2013). Function in the human connectome: Task-fMRI and individual differences in behavior. \textit{NeuroImage, 80}, 169-189. \href{https://www.ncbi.nlm.nih.gov/pmc/articles/PMC4011498}{OA} \href{https://doi.org/10.1016/j.neuroimage.2013.05.033}{DOI} \vspace{2mm}

Brakewood B, Poldrack RA.  (2013). The ethics of secondary data analysis: Considering the application of Belmont principles to the sharing of neuroimaging data. \textit{NeuroImage, 82}, 671-676. \href{https://doi.org/10.1016/j.neuroimage.2013.02.040}{DOI} \vspace{2mm}

Congdon E, Bato AA, Schonberg T, Mumford JA, Karlsgodt KH, Sabb FW, London ED, Cannon TD, Bilder RM, Poldrack RA.  (2013). Differences in neural activation as a function of risk-taking task parameters. \textit{Frontiers in Neuroscience}, 173. \href{https://doi.org/10.3389/fnins.2013.00173}{DOI} \vspace{2mm}

Davis T, Poldrack RA.  (2013). Measuring neural representations with fMRI: Practices and pitfalls. \textit{Annals of the New York Academy of Sciences, 1296}, 108-134. \href{https://doi.org/10.1111/nyas.12156}{DOI} \vspace{2mm}

Fox CR, Poldrack RA.  (2013). Prospect theory and the brain. In \textit{Neuroeconomics: Decision Making and the Brain (2nd Edition)}. \vspace{2mm}

Galván A, Schonberg T, Mumford J, Kohno M, Poldrack RA, London ED.  (2013). Greater risk sensitivity of dorsolateral prefrontal cortex in young smokers than in nonsmokers. \textit{Psychopharmacology, 229}, 345-355. \href{https://www.ncbi.nlm.nih.gov/pmc/articles/PMC3758460}{OA} \href{https://doi.org/10.1007/s00213-013-3113-x}{DOI} \vspace{2mm}

Hsieh CJ, Sustik MA, Dhillon IS, Ravikumar R, Poldrack RA.  (2013). BIG \& QUIC: Sparse Inverse Covariance Estimation for a Million Variables. \textit{Advances in Neural Information Processing Systems }. \vspace{2mm}

Hsieh CJ, Sustik MA, Dhillon IS, Ravikumar P, Poldrack RA.  (2013). BIG \&amp; QUIC: Sparse inverse covariance estimation for a million variables. \textit{Advances in Neural Information Processing Systems}. \vspace{2mm}

Koyejo O, Patel P, Ghosh J, Poldrack RA.  (2013). Learning predictive cognitive structure from fMRI using supervised topic models. \textit{Proceedings - 2013 3rd International Workshop on Pattern Recognition in Neuroimaging, PRNI 2013}, 9-12. \href{https://doi.org/10.1109/prni.2013.12}{DOI} \vspace{2mm}

Malecek NJ, Poldrack RA.  (2013). Beyond dopamine: The noradrenergic system and mental effort. \textit{Behavioral and Brain Sciences, 36}, 698-699. \href{https://doi.org/10.1017/s0140525x13001106}{DOI} \vspace{2mm}

Park M, Koyejo O, Ghosh J, Poldrack RA, Pillow JW.  (2013). Bayesian structure learning for functional neuroimaging. \textit{Sixteenth International Conference on Artificial Intelligence and Statistics (AIST ATS). }. \vspace{2mm}

Poldrack RA, Barch DM, Mitchell JP, Wager TD, Wagner AD, Devlin JT, Cumba C, Koyejo O, Milham MP.  (2013). Towards open sharing of task-based fMRI data: The OpenfMRI project. \textit{Frontiers in Neuroinformatics, 7}, 12. \href{https://doi.org/10.3389/fninf.2013.00012}{DOI} \vspace{2mm}

Poline JB, Poldrack RA.  (2013). Introduction to the special issue: Toward a new era of databasing and data sharing for neuroimaging. \textit{NeuroImage, 82}, 645-646. \href{https://doi.org/10.1016/j.neuroimage.2013.08.044}{DOI} \vspace{2mm}

White CN, Poldrack RA.  (2013). Using fMRI to Constrain Theories of Cognition. \textit{Perspectives on Psychological Science, 8}, 79-83. \href{https://doi.org/10.1177/1745691612469029}{DOI} \vspace{2mm}

Xue G, Dong Q, Chen C, Lu ZL, Mumford JA, Poldrack RA.  (2013). Complementary role of frontoparietal activity and cortical pattern similarity in successful episodic memory encoding. \textit{Cerebral Cortex, 23}, 1562-1571. \href{https://www.ncbi.nlm.nih.gov/pmc/articles/PMC3726068}{OA} \href{https://doi.org/10.1093/cercor/bhs143}{DOI} \vspace{2mm}

\subsection*{2012}Congdon E, Mumford JA, Cohen JR, Galvan A, Canli T, Poldrack RA.  (2012). Measurement and reliability of response inhibition. \textit{Frontiers in Psychology, 3}, Article 37. \href{https://doi.org/10.3389/fpsyg.2012.00037}{DOI} \vspace{2mm}

Courtney KE, Arellano R, Barkley-Levenson E, Gálvan A, Poldrack RA, Mackillop J, Jentsch J, Ray LA.  (2012). The Relationship Between Measures of Impulsivity and Alcohol Misuse: An Integrative Structural Equation Modeling Approach. \textit{Alcoholism: Clinical and Experimental Research, 36}, 923-931. \href{https://www.ncbi.nlm.nih.gov/pmc/articles/PMC3291799}{OA} \href{https://doi.org/10.1111/j.1530-0277.2011.01635.x}{DOI} \vspace{2mm}

Ghahremani DG, Lee B, Robertson CL et al. (2012). Striatal dopamine D 2/D 3 receptors mediate response inhibition and related activity in frontostriatal neural circuitry in humans. \textit{Journal of Neuroscience, 32}, 7316-7324. \href{https://www.ncbi.nlm.nih.gov/pmc/articles/PMC3517177}{OA} \href{https://doi.org/10.1523/jneurosci.4284-11.2012}{DOI} \vspace{2mm}

Ghahremani DG, Poldrack RA.  (2012). Neuroimaging and interactive memory systems. In \textit{Neuroimaging of Human Memory: Linking Cognitive Processes to Neural Systems}. \href{https://doi.org/10.1093/acprof:oso/9780199217298.003.0006}{DOI} \vspace{2mm}

Helfinstein SM, Poldrack RA.  (2012). The young and the reckless. \textit{Nature Neuroscience, 15}, 803-805. \href{https://doi.org/10.1038/nn.3116}{DOI} \vspace{2mm}

Jimura K, Poldrack RA.  (2012). Analyses of regional-average activation and multivoxel pattern information tell complementary stories. \textit{Neuropsychologia, 50}, 544-552. \href{https://doi.org/10.1016/j.neuropsychologia.2011.11.007}{DOI} \vspace{2mm}

Mumford JA, Turner BO, Ashby FG, Poldrack RA.  (2012). Deconvolving BOLD activation in event-related designs for multivoxel pattern classification analyses. \textit{NeuroImage, 59}, 2636-2643. \href{https://www.ncbi.nlm.nih.gov/pmc/articles/PMC3251697}{OA} \href{https://doi.org/10.1016/j.neuroimage.2011.08.076}{DOI} \vspace{2mm}

Poldrack RA.  (2012). The future of fMRI in cognitive neuroscience. \textit{NeuroImage, 62}, 1216-1220. \href{https://www.ncbi.nlm.nih.gov/pmc/articles/PMC4131441}{OA} \href{https://doi.org/10.1016/j.neuroimage.2011.08.007}{DOI} \vspace{2mm}

Poldrack RA, Mumford JA, Schonberg T, Kalar D, Barman B, Yarkoni T.  (2012). Discovering Relations Between Mind, Brain, and Mental Disorders Using Topic Mapping. \textit{PLoS Computational Biology, 8}, e1002707. \href{https://www.ncbi.nlm.nih.gov/pmc/articles/PMC3469446}{OA} \href{https://doi.org/10.1371/journal.pcbi.1002707}{DOI} \href{https://github.com/poldrack/LatentStructure}{Code} \vspace{2mm}

Poline JB, Poldrack RA.  (2012). Frontiers in brain imaging methods grand challenge. \textit{Frontiers in Neuroscience}, 1-2. \href{https://doi.org/10.3389/fnins.2012.00096}{DOI} \vspace{2mm}

Poline JB, Breeze JL, Ghosh S et al. (2012). Data sharing in neuroimaging research. \textit{Frontiers in Neuroinformatics, 6}, 9. \href{https://doi.org/10.3389/fninf.2012.00009}{DOI} \vspace{2mm}

Satpute AB, Mumford JA, Naliboff BD, Poldrack RA.  (2012). Human anterior and posterior hippocampus respond distinctly to state and trait anxiety. \textit{Emotion, 12}, 58-68. \href{https://doi.org/10.1037/a0026517}{DOI} \vspace{2mm}

Schonberg T, Fox CR, Mumford JA, Congdon E, Trepel C, Poldrack RA.  (2012). Decreasing ventromedial prefrontal cortex activity during sequential risk-taking: An FMRI investigation of the balloon analog risk task. \textit{Frontiers in Neuroscience}, 1-11. \href{https://doi.org/10.3389/fnins.2012.00080}{DOI} \href{https://openneuro.org/datasets/ds000001/versions/1.0.0}{Data} \vspace{2mm}

Turner BO, Mumford JA, Poldrack RA, Ashby FG.  (2012). Spatiotemporal activity estimation for multivoxel pattern analysis with rapid event-related designs. \textit{NeuroImage, 62}, 1429-1438. \href{https://www.ncbi.nlm.nih.gov/pmc/articles/PMC3408801}{OA} \href{https://doi.org/10.1016/j.neuroimage.2012.05.057}{DOI} \vspace{2mm}

Wagshal D, Knowlton BJ, Cohen JR, Poldrack RA, Bookheimer SY, Bilder RM, Fernandez VG, Asarnow RF.  (2012). Deficits in probabilistic classification learning and liability for schizophrenia. \textit{Psychiatry Research, 200}, 167-172. \href{https://www.ncbi.nlm.nih.gov/pmc/articles/PMC5332149}{OA} \href{https://doi.org/10.1016/j.psychres.2012.06.009}{DOI} \vspace{2mm}

White CN, Mumford JA, Poldrack RA.  (2012). Perceptual criteria in the human brain. \textit{Journal of Neuroscience, 32}, 16716-16724. \href{https://www.ncbi.nlm.nih.gov/pmc/articles/PMC6621768}{OA} \href{https://doi.org/10.1523/jneurosci.1744-12.2012}{DOI} \vspace{2mm}

\subsection*{2011}Cohen JR, Asarnow RF, Sabb FW, Bilder RM, Bookheimer SY, Knowlton BJ, Poldrack RA.  (2011). Decoding continuous variables from neuroimaging data: Basic and clinical applications. \textit{Frontiers in Neuroscience}, 75. \href{https://doi.org/10.3389/fnins.2011.00075}{DOI} \vspace{2mm}

Galván A, Poldrack RA, Baker CM, McGlennen KM, London ED.  (2011). Neural correlates of response inhibition and cigarette smoking in late adolescence. \textit{Neuropsychopharmacology, 36}, 970-978. \href{https://www.ncbi.nlm.nih.gov/pmc/articles/PMC3077266}{OA} \href{https://doi.org/10.1038/npp.2010.235}{DOI} \vspace{2mm}

Ghahremani DG, Tabibnia G, Monterosso J, Hellemann G, Poldrack RA, London ED.  (2011). Effect of modafinil on learning and task-related brain activity in methamphetamine-dependent and healthy individuals. \textit{Neuropsychopharmacology, 36}, 950-959. \href{https://www.ncbi.nlm.nih.gov/pmc/articles/PMC3077264}{OA} \href{https://doi.org/10.1038/npp.2010.233}{DOI} \vspace{2mm}

Lenartowicz A, Verbruggen F, Logan GD, Poldrack RA.  (2011). Inhibition-related activation in the right inferior frontal gyrus in the absence of inhibitory cues. \textit{Journal of Cognitive Neuroscience, 23}, 3388-3399. \href{https://doi.org/10.1162/jocn_a_00031}{DOI} \vspace{2mm}

Poldrack RA, Mumford JA, Nichols TE.  (2011).  \textit{Handbook of Functional MRI Data Analysis}. Cambridge: Cambridge University Press .\vspace{2mm}

Poldrack RA.  (2011). Inferring mental states from neuroimaging data: From reverse inference to large-scale decoding. \textit{Neuron, 72}, 692-697. \href{https://www.ncbi.nlm.nih.gov/pmc/articles/PMC3240863}{OA} \href{https://doi.org/10.1016/j.neuron.2011.11.001}{DOI} \vspace{2mm}

Poldrack RA, Kittur A, Kalar D, Miller E, Seppa C, Gil Y, Parker D, Sabb FW, Bilder RM.  (2011). The cognitive atlas: Toward a knowledge foundation for cognitive neuroscience. \textit{Frontiers in Neuroinformatics, 5}, 17. \href{https://doi.org/10.3389/fninf.2011.00017}{DOI} \vspace{2mm}

Rizk-Jackson A, Stoffers D, Sheldon S, Kuperman J, Dale A, Goldstein J, Corey-Bloom J, Poldrack RA, Aron AR.  (2011). Evaluating imaging biomarkers for neurodegeneration in pre-symptomatic Huntington's disease using machine learning techniques. \textit{NeuroImage, 56}, 788-796. \href{https://doi.org/10.1016/j.neuroimage.2010.04.273}{DOI} \vspace{2mm}

Schonberg T, Fox CR, Poldrack RA.  (2011). Mind the gap: Bridging economic and naturalistic risk-taking with cognitive neuroscience. \textit{Trends in Cognitive Sciences, 15}, 11-19. \href{https://www.ncbi.nlm.nih.gov/pmc/articles/PMC3014440}{OA} \href{https://doi.org/10.1016/j.tics.2010.10.002}{DOI} \vspace{2mm}

Stern JM, Caporro M, Haneef Z, Yeh HJ, Buttinelli C, Lenartowicz A, Mumford JA, Parvizi J, Poldrack RA.  (2011). Functional imaging of sleep vertex sharp transients. \textit{Clinical Neurophysiology, 122}, 1382-1386. \href{https://www.ncbi.nlm.nih.gov/pmc/articles/PMC3105179}{OA} \href{https://doi.org/10.1016/j.clinph.2010.12.049}{DOI} \vspace{2mm}

Tabibnia G, Monterosso JR, Baicy K, Aron AR, Poldrack RA, Chakrapani S, Lee B, London ED.  (2011). Different forms of self-control share a neurocognitive substrate. \textit{Journal of Neuroscience, 31}, 4805-4810. \href{https://www.ncbi.nlm.nih.gov/pmc/articles/PMC3096483}{OA} \href{https://doi.org/10.1523/jneurosci.2859-10.2011}{DOI} \vspace{2mm}

Xue G, Mei L, Chen C, Zhong-Lin L, Poldrack R, Dong Q.  (2011). Spaced learning enhances subsequent recognition memory by reducing neural repetition suppression. \textit{Journal of Cognitive Neuroscience, 23}, 1624-1633. \href{https://www.ncbi.nlm.nih.gov/pmc/articles/PMC3297428}{OA} \href{https://doi.org/10.1162/jocn.2010.21532}{DOI} \vspace{2mm}

Yarkoni T, Poldrack RA, Nichols TE, Essen DC, Wager TD.  (2011). Large-scale automated synthesis of human functional neuroimaging data. \textit{Nature Methods, 8}, 665-670. \href{https://www.ncbi.nlm.nih.gov/pmc/articles/PMC3146590}{OA} \href{https://doi.org/10.1038/nmeth.1635}{DOI} \vspace{2mm}

\subsection*{2010}Cho S, Moody TD, Fernandino L, Mumford JA, Poldrack RA, Cannon TD, Knowlton BJ, Holyoak KJ.  (2010). Common and dissociable prefrontal loci associated with component mechanisms of analogical reasoning. \textit{Cerebral Cortex, 20}, 524-533. \href{https://doi.org/10.1093/cercor/bhp121}{OA} \href{https://doi.org/10.1093/cercor/bhp121}{DOI} \vspace{2mm}

Cohen JR, Asarnow RF, Sabb FW, Bilder RM, Bookheimer SY, Knowlton BJ, Poldrack RA.  (2010). A unique adolescent response to reward prediction errors. \textit{Nature Neuroscience, 13}, 669-671. \href{https://www.ncbi.nlm.nih.gov/pmc/articles/PMC2876211}{OA} \href{https://doi.org/10.1038/nn.2558}{DOI} \vspace{2mm}

Cohen JR, Asarnow RF, Sabb FW, Bilder RM, Bookheimer SY, Knowlton BJ, Poldrack RA.  (2010). Decoding developmental differences and individual variability in response inhibition through predictive analyses across individuals. \textit{Frontiers in Human Neuroscience, 4}. \href{https://doi.org/10.3389/fnhum.2010.00047}{DOI} \vspace{2mm}

Congdon E, Mumford JA, Cohen JR, Galvan A, Aron AR, Xue G, Miller E, Poldrack RA.  (2010). Engagement of large-scale networks is related to individual differences in inhibitory control. \textit{NeuroImage, 53}, 653-663. \href{https://www.ncbi.nlm.nih.gov/pmc/articles/PMC2930099}{OA} \href{https://doi.org/10.1016/j.neuroimage.2010.06.062}{DOI} \vspace{2mm}

Congdon E, Poldrack RA, Freimer NB.  (2010). Neurocognitive phenotypes and genetic dissection of disorders of brain and behavior. \textit{Neuron, 68}, 218-230. \href{https://www.ncbi.nlm.nih.gov/pmc/articles/PMC4123421}{OA} \href{https://doi.org/10.1016/j.neuron.2010.10.007}{DOI} \vspace{2mm}

Crone EA, Poldrack RA, Durston S.  (2010). Challenges and methods in developmental neuroimaging. \textit{Human Brain Mapping, 31}, 835-837. \href{https://www.ncbi.nlm.nih.gov/pmc/articles/PMC6870689}{OA} \href{https://doi.org/10.1002/hbm.21053}{DOI} \vspace{2mm}

Ghahremani DG, Monterosso J, Jentsch JD, Bilder RM, Poldrack RA.  (2010). Neural components underlying behavioral flexibility in human reversal learning. \textit{Cerebral Cortex, 20}, 1843-1852. \href{https://www.ncbi.nlm.nih.gov/pmc/articles/PMC2901019}{OA} \href{https://doi.org/10.1093/cercor/bhp247}{DOI} \vspace{2mm}

Kenner NM, Mumford JA, Hommer RE, Skup M, Leibenluft E, Poldrack RA.  (2010). Inhibitory motor control in response stopping and response switching. \textit{Journal of Neuroscience, 30}, 8512-8518. \href{https://www.ncbi.nlm.nih.gov/pmc/articles/PMC2905623}{OA} \href{https://doi.org/10.1523/jneurosci.1096-10.2010}{DOI} \vspace{2mm}

Kriegeskorte N, Lindquist MA, Nichols TE, Poldrack RA, Vul E.  (2010). Everything you never wanted to know about circular analysis, but were afraid to ask. \textit{Journal of Cerebral Blood Flow and Metabolism, 30}, 1551-1557. \href{https://www.ncbi.nlm.nih.gov/pmc/articles/PMC2949251}{OA} \href{https://doi.org/10.1038/jcbfm.2010.86}{DOI} \vspace{2mm}

Lenartowicz A, Poldrack RA.  (2010). Brain Imaging. In \textit{Encyclopedia of Behavioral Neuroscience}, 187-193. \href{https://doi.org/10.1016/b978-0-08-045396-5.00052-x}{DOI} \vspace{2mm}

Lenartowicz A, Kalar DJ, Congdon E, Poldrack RA.  (2010). Towards an Ontology of Cognitive Control. \textit{Topics in Cognitive Science, 2}, 678-692. \href{https://doi.org/10.1111/j.1756-8765.2010.01100.x}{DOI} \vspace{2mm}

Mumford JA, Horvath S, Oldham MC, Langfelder P, Geschwind DH, Poldrack RA.  (2010). Detecting network modules in fMRI time series: A weighted network analysis approach. \textit{NeuroImage, 52}, 1465-1476. \href{https://www.ncbi.nlm.nih.gov/pmc/articles/PMC3632300}{OA} \href{https://doi.org/10.1016/j.neuroimage.2010.05.047}{DOI} \vspace{2mm}

Poldrack RA, Carr V, Foerde K.  (2010). Flexibility and generalization in memory systems. In \textit{Generalization of Knowledge: Multidisciplinary perspectives }, 53-70. \vspace{2mm}

Poldrack RA.  (2010). Subtraction and beyond: The logic of experimental designs for neuroimaging. In \textit{Foundational Issues in Human Brain Mapping}, 147-160. \href{https://doi.org/10.7551/mitpress/9780262014021.001.0001}{DOI} \vspace{2mm}

Poldrack RA, Mumford JA.  (2010). On the proper role of non-independent ROI analysis: A commentary on Vul and Kanwisher. In \textit{Foundational Issues in Human Brain Mapping}, 93-96. \href{https://doi.org/10.7551/mitpress/9780262014021.001.0001}{DOI} \vspace{2mm}

Poldrack RA.  (2010). Mapping mental function to brain Structure: How can cognitive Neuroimaging Succeed?. \textit{Perspectives on Psychological Science, 5}, 753-761. \href{https://doi.org/10.1177/1745691610388777}{DOI} \vspace{2mm}

Poldrack RA.  (2010). Interpreting developmental changes in neuroimaging signals. \textit{Human Brain Mapping, 31}, 872-878. \href{https://www.ncbi.nlm.nih.gov/pmc/articles/PMC6870770}{OA} \href{https://doi.org/10.1002/hbm.21039}{DOI} \vspace{2mm}

Ramsey JD, Hanson SJ, Hanson C, Halchenko YO, Poldrack RA, Glymour C.  (2010). Six problems for causal inference from fMRI. \textit{NeuroImage, 49}, 1545-1558. \href{https://doi.org/10.1016/j.neuroimage.2009.08.065}{DOI} \vspace{2mm}

Zeeland AA, Abrahams BS, Alvarez-Retuerto AI et al. (2010). Altered functional connectivity in frontal lobe circuits is associated with variation in the autism risk gene CNTNAP. \textit{Science Translational Medicine, 2}, 56ra80. \href{https://www.ncbi.nlm.nih.gov/pmc/articles/PMC3065863}{OA} \href{https://doi.org/10.1126/scitranslmed.3001344}{DOI} \vspace{2mm}

Zeeland AA, Dapretto M, Ghahremani DG, Poldrack RA, Bookheimer SY.  (2010). Reward processing in autism. \textit{Autism Research, 3}, 53-67. \href{https://www.ncbi.nlm.nih.gov/pmc/articles/PMC3076289}{OA} \href{https://doi.org/10.1002/aur.122}{DOI} \vspace{2mm}

Xue G, Mei L, Chen C, Lu ZL, Poldrack RA, Dong Q.  (2010). Facilitating memory for novel characters by reducing neural repetition suppression in the left fusiform cortex. \textit{PLoS ONE, 5}, e13204. \href{https://www.ncbi.nlm.nih.gov/pmc/articles/PMC2950859}{OA} \href{https://doi.org/10.1371/journal.pone.0013204}{DOI} \vspace{2mm}

Xue G, Dong Q, Chen C, Lu Z, Mumford JA, Poldrack RA.  (2010). Greater neural pattern similarity across repetitions is associated with better memory. \textit{Science, 330}, 97-101. \href{https://www.ncbi.nlm.nih.gov/pmc/articles/PMC2952039}{OA} \href{https://doi.org/10.1126/science.1193125}{DOI} \vspace{2mm}

Yarkoni T, Poldrack RA, Essen DC, Wager TD.  (2010). Cognitive neuroscience 2.0: Building a cumulative science of human brain function. \textit{Trends in Cognitive Sciences, 14}, 489-496. \href{https://www.ncbi.nlm.nih.gov/pmc/articles/PMC2963679}{OA} \href{https://doi.org/10.1016/j.tics.2010.08.004}{DOI} \vspace{2mm}

\subsection*{2009}Aron AR, Wise SP, Poldrack RA.  (2009). Role of the basal ganglia in cognition. In \textit{The New Encyclopedia of Neuroscience (Volume 2)}, 1069-1077. \vspace{2mm}

Aron AR, Poldrack RA, Wise SP.  (2009). Cognition: Basal Ganglia Role. In \textit{Encyclopedia of Neuroscience}, 1069-1077. \href{https://doi.org/10.1016/b978-008045046-9.00410-1}{DOI} \vspace{2mm}

Barch DM, Braver TS, Carter CS, Poldrack RA, Robbins TW.  (2009). CNTRICS final task selection: Executive control. \textit{Schizophrenia Bulletin, 35}, 115-135. \href{https://www.ncbi.nlm.nih.gov/pmc/articles/PMC2643948}{OA} \href{https://doi.org/10.1093/schbul/sbn154}{DOI} \vspace{2mm}

Bilder RM, Poldrack RA, Parker DS et al. (2009). Cognitive phenomics. In \textit{Handbook of Neuropsychology of Mental Disorders }. \vspace{2mm}

Bilder RM, Sabb FW, Cannon TD, London ED, Jentsch JD, Parker DS, Poldrack RA, Evans C, Freimer NB.  (2009). Phenomics: The systematic study of phenotypes on a genome-wide scale. \textit{Neuroscience, 164}, 30-42. \href{https://www.ncbi.nlm.nih.gov/pmc/articles/PMC2760679}{OA} \href{https://doi.org/10.1016/j.neuroscience.2009.01.027}{DOI} \vspace{2mm}

Bilder RM, Sabb FW, Parker DS, Kalar D, Chu WW, Fox J, Freimer NB, Poldrack RA.  (2009). Cognitive ontologies for neuropsychiatric phenomics research. \textit{Cognitive Neuropsychiatry, 14}, 419-450. \href{https://www.ncbi.nlm.nih.gov/pmc/articles/PMC2752634}{OA} \href{https://doi.org/10.1080/13546800902787180}{DOI} \vspace{2mm}

Foerde K, Poldrack RA.  (2009). Procedural Learning in Humans. In \textit{Encyclopedia of Neuroscience}, 1083-1091. \href{https://doi.org/10.1016/b978-008045046-9.00783-x}{DOI} \vspace{2mm}

Glimcher P, Fehr E, Camerer C, Poldrack RA (Eds.).  (2009).  \textit{Neuroeconomics}. San Diego: Academic Press .\href{https://doi.org/10.1016/b978-0-12-374176-9.x0001-2}{DOI} \vspace{2mm}

Glimcher PW, Camerer CF, Fehr E, Poldrack RA.  (2009). Introduction: A brief history of neuroeconomics. In \textit{Neuroeconomics}, 1-12. \href{https://doi.org/10.1016/b978-0-12-374176-9.00001-4}{DOI} \vspace{2mm}

Lee B, London ED, Poldrack RA et al. (2009). Striatal dopamine D2/D3 receptor availability is reduced in methamphetamine dependence and is linked to impulsivity. \textit{Journal of Neuroscience, 29}, 14734-14740. \href{https://www.ncbi.nlm.nih.gov/pmc/articles/PMC2822639}{OA} \href{https://doi.org/10.1523/jneurosci.3765-09.2009}{DOI} \vspace{2mm}

Poldrack RA, Halchenko YO, Hanson SJ.  (2009). Decoding the large-scale structure of brain function by classifying mental states across individuals. \textit{Psychological Science, 20}, 1364-1372. \href{https://www.ncbi.nlm.nih.gov/pmc/articles/PMC2935493}{OA} \href{https://doi.org/10.1111/j.1467-9280.2009.02460.x}{DOI} \vspace{2mm}

Poldrack RA, Mumford JA.  (2009). Independence in ROI analysis: Where is the voodoo?. \textit{Social Cognitive and Affective Neuroscience, 4}, 208-213. \href{https://www.ncbi.nlm.nih.gov/pmc/articles/PMC2686233}{OA} \href{https://doi.org/10.1093/scan/nsp011}{DOI} \vspace{2mm}

Sabb FW, Burggren AC, Higier RG et al. (2009). Challenges in phenotype definition in the whole-genome era: Multivariate models of memory and intelligence. \textit{Neuroscience, 164}, 88-107. \href{https://www.ncbi.nlm.nih.gov/pmc/articles/PMC2766544}{OA} \href{https://doi.org/10.1016/j.neuroscience.2009.05.013}{DOI} \vspace{2mm}

Thompson PM, Miller MI, Poldrack RA, Nichols TE, Taylor JE, Worsley KJ, Ratnanather JT.  (2009). Special issue on mathematics in brain imaging.. \textit{NeuroImage, 45}. \href{https://doi.org/10.1016/j.neuroimage.2008.10.033}{DOI} \vspace{2mm}

Horn JD, Poldrack RA.  (2009). Functional MRI at the crossroads. \textit{International Journal of Psychophysiology, 73}, 3-9. \href{https://www.ncbi.nlm.nih.gov/pmc/articles/PMC2747289}{OA} \href{https://doi.org/10.1016/j.ijpsycho.2008.11.003}{DOI} \vspace{2mm}

\subsection*{2008}Cohen JR, Poldrack RA.  (2008). Automaticity in motor sequence learning does not impair response inhibition. \textit{Psychonomic Bulletin and Review, 15}, 108-115. \href{https://doi.org/10.3758/pbr.15.1.108}{OA} \href{https://doi.org/10.3758/pbr.15.1.108}{DOI} \vspace{2mm}

Foerde K, Poldrack RA, Knowlton BJ et al. (2008). Selective Corticostriatal Dysfunction in Schizophrenia: Examination of Motor and Cognitive Skill Learning. \textit{Neuropsychology, 22}, 100-109. \href{https://doi.org/10.1037/0894-4105.22.1.100}{DOI} \vspace{2mm}

Gluck MA, Poldrack RA, Kéri S.  (2008). The cognitive neuroscience of category learning. \textit{Neuroscience and Biobehavioral Reviews, 32}, 193-196. \href{https://doi.org/10.1016/j.neubiorev.2007.11.002}{DOI} \vspace{2mm}

Karlsgodt KH, Erp TG, Poldrack RA, Bearden CE, Nuechterlein KH, Cannon TD.  (2008). Diffusion Tensor Imaging of the Superior Longitudinal Fasciculus and Working Memory in Recent-Onset Schizophrenia. \textit{Biological Psychiatry, 63}, 512-518. \href{https://doi.org/10.1016/j.biopsych.2007.06.017}{DOI} \vspace{2mm}

Poldrack RA.  (2008). The role of fMRI in Cognitive Neuroscience: where do we stand?. \textit{Current Opinion in Neurobiology, 18}, 223-227. \href{https://doi.org/10.1016/j.conb.2008.07.006}{DOI} \vspace{2mm}

Poldrack RA, Wagner AD.  (2008). The interface between neuroscience and psychological science: Introduction to the special issue. \textit{Current Directions in Psychological Science, 17}, 61. \href{https://doi.org/10.1111/j.1467-8721.2008.00549.x}{DOI} \vspace{2mm}

Poldrack RA, Fletcher PC, Henson RN, Worsley KJ, Brett M, Nichols TE.  (2008). Guidelines for reporting an fMRI study. \textit{NeuroImage, 40}, 409-414. \href{https://www.ncbi.nlm.nih.gov/pmc/articles/PMC2287206}{OA} \href{https://doi.org/10.1016/j.neuroimage.2007.11.048}{DOI} \vspace{2mm}

Poldrack RA, Foerde K.  (2008). Category learning and the memory systems debate. \textit{Neuroscience and Biobehavioral Reviews, 32}, 197-205. \href{https://doi.org/10.1016/j.neubiorev.2007.07.007}{DOI} \vspace{2mm}

Raizada RD, Poldrack RA.  (2008). Challenge-driven attention: Interacting frontal and brainstem systems. \textit{Frontiers in Human Neuroscience, 1}, 3. \href{https://doi.org/10.3389/neuro.09.003.2007}{DOI} \vspace{2mm}

Shattuck DW, Mirza M, Adisetiyo V, Hojatkashani C, Salamon G, Narr KL, Poldrack RA, Bilder RM, Toga AW.  (2008). Construction of a 3D probabilistic atlas of human cortical structures. \textit{NeuroImage, 39}, 1064-1080. \href{https://www.ncbi.nlm.nih.gov/pmc/articles/PMC2757616}{OA} \href{https://doi.org/10.1016/j.neuroimage.2007.09.031}{DOI} \vspace{2mm}

Tohka J, Foerde K, Aron AR, Tom SM, Toga AW, Poldrack RA.  (2008). Automatic independent component labeling for artifact removal in fMRI. \textit{NeuroImage, 39}, 1227-1245. \href{https://www.ncbi.nlm.nih.gov/pmc/articles/PMC2374836}{OA} \href{https://doi.org/10.1016/j.neuroimage.2007.10.013}{DOI} \vspace{2mm}

Xue G, Ghahremani DG, Poldrack RA.  (2008). Neural substrates for reversing stimulus-outcome and stimulus-response associations. \textit{Journal of Neuroscience, 28}, 11196-11204. \href{https://www.ncbi.nlm.nih.gov/pmc/articles/PMC6671509}{OA} \href{https://doi.org/10.1523/jneurosci.4001-08.2008}{DOI} \vspace{2mm}

Xue G, Aron AR, Poldrack RA.  (2008). Common neural substrates for inhibition of spoken and manual responses. \textit{Cerebral Cortex, 18}, 1923-1932. \href{https://doi.org/10.1093/cercor/bhm220}{OA} \href{https://doi.org/10.1093/cercor/bhm220}{DOI} \href{https://openneuro.org/datasets/ds000007/versions/00001}{Data} \vspace{2mm}

\subsection*{2007}Aron AR, Behrens TE, Smith S, Frank MJ, Poldrack RA.  (2007). Triangulating a cognitive control network using diffusion-weighted Magnetic Resonance Imaging (MRI) and functional MRI. \textit{Journal of Neuroscience, 27}, 3743-3752. \href{https://www.ncbi.nlm.nih.gov/pmc/articles/PMC6672420}{OA} \href{https://doi.org/10.1523/jneurosci.0519-07.2007}{DOI} \vspace{2mm}

Devlin JT, Poldrack RA.  (2007). In praise of tedious anatomy. \textit{NeuroImage, 37}, 1033-1041. \href{https://www.ncbi.nlm.nih.gov/pmc/articles/PMC1986635}{OA} \href{https://doi.org/10.1016/j.neuroimage.2006.09.055}{DOI} \vspace{2mm}

Foerde K, Poldrack RA, Knowlton BJ.  (2007). Secondary-task effects on classification learning. \textit{Memory and Cognition, 35}, 864-874. \href{https://doi.org/10.3758/bf03193461}{DOI} \vspace{2mm}

Mumford JA, Poldrack RA.  (2007). Modeling group fMRI data.. \textit{Social cognitive and affective neuroscience, 2}, 251-257. \href{https://www.ncbi.nlm.nih.gov/pmc/articles/PMC2569805}{OA} \href{https://doi.org/10.1093/scan/nsm019}{DOI} \vspace{2mm}

Poldrack RA, Devlin JT.  (2007). On the fundamental role of anatomy in functional imaging: Reply to commentaries on "In praise of tedious anatomy". \textit{NeuroImage, 37}, 1066-1068. \href{https://doi.org/10.1016/j.neuroimage.2007.06.019}{DOI} \vspace{2mm}

Poldrack RA.  (2007). Region of interest analysis for fMRI. \textit{Social Cognitive and Affective Neuroscience, 2}, 67-70. \href{https://www.ncbi.nlm.nih.gov/pmc/articles/PMC2555436}{OA} \href{https://doi.org/10.1093/scan/nsm006}{DOI} \vspace{2mm}

Raizada RD, Poldrack RA.  (2007). Selective Amplification of Stimulus Differences during Categorical Processing of Speech. \textit{Neuron, 56}, 726-740. \href{https://doi.org/10.1016/j.neuron.2007.11.001}{OA} \href{https://doi.org/10.1016/j.neuron.2007.11.001}{DOI} \vspace{2mm}

Thermenos HW, Seidman LJ, Poldrack RA, Peace NK, Koch JK, Faraone SV, Tsuang MT.  (2007). Elaborative Verbal Encoding and Altered Anterior Parahippocampal Activation in Adolescents and Young Adults at Genetic Risk for Schizophrenia Using fMRI. \textit{Biological Psychiatry, 61}, 564-574. \href{https://doi.org/10.1016/j.biopsych.2006.04.044}{DOI} \vspace{2mm}

Tom SM, Fox CR, Trepel C, Poldrack RA.  (2007). The neural basis of loss aversion in decision-making under risk. \textit{Science, 315}, 515-518. \href{https://doi.org/10.1126/science.1134239}{DOI} \href{https://openneuro.org/datasets/ds000008/versions/00001}{Data} \vspace{2mm}

Xue G, Poldrack RA.  (2007). The neural substrates of visual perceptual learning of words: Implications for the visual word form area hypothesis. \textit{Journal of Cognitive Neuroscience, 19}, 1643-1655. \href{https://doi.org/10.1162/jocn.2007.19.10.1643}{DOI} \vspace{2mm}

\subsection*{2006}Aron AR, Poldrack RA.  (2006). Cortical and subcortical contributions to stop signal response inhibition: Role of the subthalamic nucleus. \textit{Journal of Neuroscience, 26}, 2424-2433. \href{https://www.ncbi.nlm.nih.gov/pmc/articles/PMC6793670}{OA} \href{https://doi.org/10.1523/jneurosci.4682-05.2006}{DOI} \vspace{2mm}

Aron AR, Gluck MA, Poldrack RA.  (2006). Long-term test-retest reliability of functional MRI in a classification learning task. \textit{NeuroImage, 29}, 1000-1006. \href{https://www.ncbi.nlm.nih.gov/pmc/articles/PMC1630684}{OA} \href{https://doi.org/10.1016/j.neuroimage.2005.08.010}{DOI} \href{https://openneuro.org/datasets/ds000017/versions/00001}{Data} \vspace{2mm}

Foerde K, Knowlton BJ, Poldrack RA.  (2006). Modulation of competing memory systems by distraction. \textit{Proceedings of the National Academy of Sciences of the United States of America, 103}, 11778-11783. \href{https://www.ncbi.nlm.nih.gov/pmc/articles/PMC1544246}{OA} \href{https://doi.org/10.1073/pnas.0602659103}{DOI} \href{https://openneuro.org/datasets/ds000011/versions/00001}{Data} \vspace{2mm}

Poldrack RA, Willingham DB.  (2006). Functional neuroimaging of skill learning. In \textit{Handbook of Neuroimaging of Cognition, 2nd Edition }, 113-148. \vspace{2mm}

Poldrack RA.  (2006). Can cognitive processes be inferred from neuroimaging data?. \textit{Trends in Cognitive Sciences, 10}, 59-63. \href{https://doi.org/10.1016/j.tics.2005.12.004}{DOI} \vspace{2mm}

Rodriguez PF, Aron AR, Poldrack RA.  (2006). Ventral-striatal/nucleus-accumbens sensitivity to prediction errors during classification learning. \textit{Human Brain Mapping, 27}, 306-313. \href{https://www.ncbi.nlm.nih.gov/pmc/articles/PMC6871483}{OA} \href{https://doi.org/10.1002/hbm.20186}{DOI} \vspace{2mm}

Seidman LJ, Thermenos HW, Poldrack RA, Peace NK, Koch JK, Faraone SV, Tsuang MT.  (2006). Altered brain activation in dorsolateral prefrontal cortex in adolescents and young adults at genetic risk for schizophrenia: An fMRI study of working memory. \textit{Schizophrenia Research, 85}, 58-72. \href{https://doi.org/10.1016/j.schres.2006.03.019}{DOI} \vspace{2mm}

Yu P, Han X, Ségonne F, Liu AK, Poldrack RA, Golland P, Fischl B.  (2006). Shape-based discrimination and classification of cortical surfaces. \textit{Proceedings - International Conference on Pattern Recognition, 3}, 445-448. \href{https://doi.org/10.1109/icpr.2006.1052}{DOI} \vspace{2mm}

\subsection*{2005}Aron AR, Poldrack RA.  (2005). The cognitive neuroscience of response inhibition: Relevance for genetic research in attention-deficit/hyperactivity disorder. \textit{Biological Psychiatry, 57}, 1285-1292. \href{https://doi.org/10.1016/j.biopsych.2004.10.026}{DOI} \vspace{2mm}

Badre D, Poldrack RA, Paré-Blagoev EJ, Insler RZ, Wagner AD.  (2005). Dissociable controlled retrieval and generalized selection mechanisms in ventrolateral prefrontal cortex. \textit{Neuron, 47}, 907-918. \href{https://doi.org/10.1016/j.neuron.2005.07.023}{OA} \href{https://doi.org/10.1016/j.neuron.2005.07.023}{DOI} \vspace{2mm}

Goldstein JM, Jerram M, Poldrack R, Ahern T, Kennedy DN, Seidman LJ, Makris N.  (2005). Hormonal cycle modulates arousal circuitry in women using functional magnetic resonance imaging. \textit{Journal of Neuroscience, 25}, 9309-9316. \href{https://www.ncbi.nlm.nih.gov/pmc/articles/PMC6725775}{OA} \href{https://doi.org/10.1523/jneurosci.2239-05.2005}{DOI} \vspace{2mm}

Goldstein JM, Poldrack R, Breiter HC, Makris N, Goodman JM, Jerram M, Anagnoson R, Tsuang MT, Seidman LJ.  (2005). Sex differences in prefrontal cortical brain activity during fMRI of auditory verbal working memory. \textit{Neuropsychology, 19}, 509-519. \href{https://doi.org/10.1037/0894-4105.19.4.509}{DOI} \vspace{2mm}

Katzir T, Misra M, Poldrack RA.  (2005). Imaging phonology without print: Assessing the neural correlates of phonemic awareness using fMRI. \textit{NeuroImage, 27}, 106-115. \href{https://doi.org/10.1016/j.neuroimage.2005.04.013}{DOI} \vspace{2mm}

Poldrack RA, Sabb FW, Foerde K, Tom SM, Asarnow RF, Bookheimer SY, Knowlton BJ.  (2005). The neural correlates of motor skill automaticity. \textit{Journal of Neuroscience, 25}, 5356-5364. \href{https://www.ncbi.nlm.nih.gov/pmc/articles/PMC6725010}{OA} \href{https://doi.org/10.1523/jneurosci.3880-04.2005}{DOI} \vspace{2mm}

Stone WS, Thermenos HW, Tarbox SI, Poldrack RA, Seidman LJ.  (2005). Medial temporal and prefrontal lobe activation during verbal encoding following glucose ingestion in schizophrenia: A pilot fMRI study. \textit{Neurobiology of Learning and Memory, 83}, 54-64. \href{https://doi.org/10.1016/j.nlm.2004.07.009}{DOI} \vspace{2mm}

Thermenos HW, Goldstein JM, Buka SL, Poldrack RA, Koch JK, Tsuang MT, Seidman LJ.  (2005). The effect of working memory performance on functional MRI in schizophrenia. \textit{Schizophrenia Research, 74}, 179-194. \href{https://doi.org/10.1016/j.schres.2004.07.021}{DOI} \vspace{2mm}

Trepel C, Fox CR, Poldrack RA.  (2005). Prospect theory on the brain? Toward a cognitive neuroscience of decision under risk. \textit{Cognitive Brain Research, 23}, 34-50. \href{https://doi.org/10.1016/j.cogbrainres.2005.01.016}{DOI} \vspace{2mm}

Valera EM, Faraone SV, Biederman J, Poldrack RA, Seidman LJ.  (2005). Functional neuroanatomy of working memory in adults with attention-deficit/hyperactivity disorder. \textit{Biological Psychiatry, 57}, 439-447. \href{https://doi.org/10.1016/j.biopsych.2004.11.034}{DOI} \vspace{2mm}

\subsection*{2004}Aron AR, Shohamy D, Clark J, Myers C, Gluck MA, Poldrack RA.  (2004). Human midbrain sensitivity to cognitive feedback and uncertainty during classification learning. \textit{Journal of Neurophysiology, 92}, 1144-1152. \href{https://doi.org/10.1152/jn.01209.2003}{DOI} \vspace{2mm}

Aron AR, Robbins TW, Poldrack RA.  (2004). Inhibition and the right inferior frontal cortex. \textit{Trends in Cognitive Sciences, 8}, 170-177. \href{https://doi.org/10.1016/j.tics.2004.02.010}{DOI} \vspace{2mm}

Misra M, Katzir T, Wolf M, Poldrack RA.  (2004). Neural systems for rapid automatized naming in skilled readers: Unraveling the RAN-reading relationship. \textit{Scientific Studies of Reading, 8}, 241-256. \href{https://doi.org/10.1207/s1532799xssr0803_4}{DOI} \vspace{2mm}

Poldrack RA, Rodriguez P.  (2004). How do memory systems interact? Evidence from human classification learning. \textit{Neurobiology of Learning and Memory, 82}, 324-332. \href{https://doi.org/10.1016/j.nlm.2004.05.003}{DOI} \vspace{2mm}

Poldrack RA, Wagner AD.  (2004). What can neuroimaging tell us about the mind? Insights from prefrontal cortex. \textit{Current Directions in Psychological Science, 13}, 177-181. \href{https://doi.org/10.1111/j.0963-7214.2004.00302.x}{DOI} \vspace{2mm}

Poldrack RA, Sandak R.  (2004). Introduction to this special issue: The cognitive neuroscience of reading. \textit{Scientific Studies of Reading, 8}, 199-202. \href{https://doi.org/10.1207/s1532799xssr0803_1}{DOI} \vspace{2mm}

Shohamy D, Myers CE, Grossman S, Sage J, Gluck MA, Poldrack RA.  (2004). Cortico-striatal contributions to feedback-based learning: Converging data from neuroimaging and neuropsychology. \textit{Brain, 127}, 851-859. \href{https://doi.org/10.1093/brain/awh100}{OA} \href{https://doi.org/10.1093/brain/awh100}{DOI} \vspace{2mm}

Thermenos HW, Seidman LJ, Breiter H, Goldstein JM, Goodman JM, Poldrack R, Faraone SV, Tsuang MT.  (2004). Functional magnetic resonance imaging during auditory verbal working memory in nonpsychotic relatives of persons with schizophrenia: A pilot study. \textit{Biological Psychiatry, 55}, 490-500. \href{https://doi.org/10.1016/j.biopsych.2003.11.014}{DOI} \vspace{2mm}

Thompson PM, Miller MI, Ratnanather JT, Poldrack RA, Nichols TE.  (2004). Preface to the special issue. \textit{NeuroImage, 23}. \href{https://doi.org/10.1016/j.neuroimage.2004.07.009}{DOI} \vspace{2mm}

\subsection*{2003}Boas DA, Strangman G, Culver JP, Hoge RD, Jasdzewski G, Poldrack RA, Rosen BR, Mandeville JB.  (2003). Can the cerebral metabolic rate of oxygen be estimated with near-infrared spectroscopy?. \textit{Physics in Medicine and Biology, 48}, 2405-2418. \href{https://doi.org/10.1088/0031-9155/48/15/311}{DOI} \vspace{2mm}

Jasdzewski G, Strangman G, Wagner J, Kwong KK, Poldrack RA, Boas DA.  (2003). Differences in the hemodynamic response to event-related motor and visual paradigms as measured by near-infrared spectroscopy. \textit{NeuroImage, 20}, 479-488. \href{https://doi.org/10.1016/s1053-8119(03)00311-2}{DOI} \vspace{2mm}

Lai I, Gollub R, Hoge R, Greve D, Vangel M, Poldrack R, Greenberg JE.  (2003). Teaching statistical analysis of fMRI data. \textit{ASEE Annual Conference Proceedings}, 6019-6029. \vspace{2mm}

Misra M, Katzir T, Wolf M, Poldrack RA.  (2003). Neural systems for rapid automatized naming identified using fMRI. \textit{Scientific Studies of Reading, 8.0}, 241-256. \vspace{2mm}

Poldrack RA, Sandak R.  (2003). Introduction to special issue: The cognitive neuroscience of reading. \textit{Scientific Studies of Reading, 8.0}, 199-202. \vspace{2mm}

Poldrack RA, Rodriguez P.  (2003). Sequence learning: What's the hippocampus to do?. \textit{Neuron, 37}, 891-893. \href{https://doi.org/10.1016/s0896-6273(03)00159-4}{OA} \href{https://doi.org/10.1016/s0896-6273(03)00159-4}{DOI} \vspace{2mm}

Poldrack RA, Packard MG.  (2003). Competition among multiple memory systems: Converging evidence from animal and human brain studies. \textit{Neuropsychologia, 41}, 245-251. \href{https://doi.org/10.1016/s0028-3932(02)00157-4}{DOI} \vspace{2mm}

Sperling R, Chua E, Cocchiarella A, Rand-Giovannetti E, Poldrack R, Schacter DL, Albert M.  (2003). Putting names to faces: Successful encoding of associative memories activates the anterior hippocampal formation. \textit{NeuroImage, 20}, 1400-1410. \href{https://www.ncbi.nlm.nih.gov/pmc/articles/PMC3230827}{OA} \href{https://doi.org/10.1016/s1053-8119(03)00391-4}{DOI} \vspace{2mm}

Temple E, Deutsch GK, Poldrack RA, Miller SL, Tallal P, Merzenich MM, Gabrieli JD.  (2003). Neural deficits in children with dyslexia ameliorated by behavioral remediation: Evidence from functional MRI. \textit{Proceedings of the National Academy of Sciences of the United States of America, 100}, 2860-2865. \href{https://www.ncbi.nlm.nih.gov/pmc/articles/PMC151431}{OA} \href{https://doi.org/10.1073/pnas.0030098100}{DOI} \vspace{2mm}

\subsection*{2002}Boas DA, Jasdzewski G, Strangman G, Culver JP, Poldrack R.  (2002). Modeling of the Hemodynamic Response Function for Event Related Motor and Visual Stimuli as Measured by Near Infrared Spectroscopy. \textit{Optics InfoBase Conference Papers}, 307-309. \vspace{2mm}

Golby AJ, Poldrack RA, Illes J, Chen D, Desmond JE, Gabrieli JD.  (2002). Memory lateralization in medial temporal lobe epilepsy assessed by functional MRI. \textit{Epilepsia, 43}, 855-863. \href{https://doi.org/10.1046/j.1528-1157.2002.20501.x}{DOI} \vspace{2mm}

Poldrack RA, Paré-Blagoev EJ, Grant PE.  (2002). Pediatric functional magnetic resonance imaging: Progress and challenges. \textit{Topics in Magnetic Resonance Imaging, 13}, 61-70. \href{https://doi.org/10.1097/00002142-200202000-00005}{DOI} \vspace{2mm}

Poldrack RA.  (2002). Neural systems for perceptual skill learning.. \textit{Behavioral and cognitive neuroscience reviews, 1}, 76-83. \href{https://doi.org/10.1177/1534582302001001005}{DOI} \vspace{2mm}

\subsection*{2001}Golby AJ, Poldrack RA, Brewer JB, Spencer D, Desmond JE, Aron AP, Gabrieli JD.  (2001). Material-specific lateralization in the medial temporal lobe and prefrontal cortex during memory encoding. \textit{Brain, 124}, 1841-1854. \href{https://doi.org/10.1093/brain/124.9.1841}{OA} \href{https://doi.org/10.1093/brain/124.9.1841}{DOI} \vspace{2mm}

Poldrack RA.  (2001). A Structural Basis for Developmental Dyslexia:Evidence from Diffusion Tensor Imaging. In \textit{Dyslexia, Fluency and the Brain}, 213-234. \vspace{2mm}

Poldrack RA, Clark J, Paré-Blagoev EJ, Shohamy D, Moyano J, Myers C, Gluck MA.  (2001). Interactive memory systems in the human brain. \textit{Nature, 414}, 546-550. \href{https://doi.org/10.1038/35107080}{DOI} \href{https://openneuro.org/datasets/ds000052/versions/00001}{Data} \vspace{2mm}

Poldrack RA, Temple E, Protopapas A, Nagarajan S, Tallal P, Merzenich M, Gabrieli JD.  (2001). Relations between the neural bases of dynamic auditory processing and phonological processing: Evidence from fMRI. \textit{Journal of Cognitive Neuroscience, 13}, 687-697. \href{https://doi.org/10.1162/089892901750363235}{DOI} \vspace{2mm}

Poldrack RA, Gabrieli JD.  (2001). Characterizing the neural mechanisms of skill learning and repetition priming evidence from mirror reading. \textit{Brain, 124}, 67-82. \href{https://doi.org/10.1093/brain/124.1.67}{OA} \href{https://doi.org/10.1093/brain/124.1.67}{DOI} \vspace{2mm}

Temple E, Poldrack RA, Salidis J, Deutsch GK, Tallal P, Merzenich MM, Gabrieli JD.  (2001). Disrupted neural responses to phonological and orthographic processing in dyslexic children: An fMRI study. \textit{NeuroReport, 12}, 299-307. \href{https://doi.org/10.1097/00001756-200102120-00024}{DOI} \vspace{2mm}

Vuilleumier P, Sagiv N, Hazeltine E, Poldrack RA, Swick D, Rafal RD, Gabrieli JD.  (2001). Neural fate of seen and unseen faces in visuospatial neglect: A combined event-related functional MRI and event-related potential study. \textit{Proceedings of the National Academy of Sciences of the United States of America, 98}, 3495-3500. \href{https://www.ncbi.nlm.nih.gov/pmc/articles/PMC30681}{OA} \href{https://doi.org/10.1073/pnas.051436898}{DOI} \vspace{2mm}

Wagner AD, Paré-Blagoev EJ, Clark J, Poldrack RA.  (2001). Recovering meaning: Left prefrontal cortex guides controlled semantic retrieval. \textit{Neuron, 31}, 329-338. \href{https://doi.org/10.1016/s0896-6273(01)00359-2}{OA} \href{https://doi.org/10.1016/s0896-6273(01)00359-2}{DOI} \vspace{2mm}

\subsection*{2000}Hazeltine E, Poldrack R, Gabrieli JD.  (2000). Neural activation during response competition. \textit{Journal of Cognitive Neuroscience, 12}, 118-129. \href{https://doi.org/10.1162/089892900563984}{DOI} \vspace{2mm}

Klingberg T, Hedehus M, Temple E, Salz T, Gabrieli JD, Moseley ME, Poldrack RA.  (2000). Microstructure of temporo-parietal white matter as a basis for reading ability: Evidence from diffusion tensor magnetic resonance imaging. \textit{Neuron, 25}, 493-500. \href{https://doi.org/10.1016/s0896-6273(00)80911-3}{OA} \href{https://doi.org/10.1016/s0896-6273(00)80911-3}{DOI} \vspace{2mm}

Poldrack RA.  (2000). Imaging brain plasticity: Conceptual and methodological issues - A theoretical review. \textit{NeuroImage, 12}, 1-13. \href{https://doi.org/10.1006/nimg.2000.0596}{DOI} \vspace{2mm}

Seger CA, Poldrack RA, Prabhakaran V, Zhao M, Glover GH, Gabrieli JD.  (2000). Hemispheric asymmetries and individual differences in visual concept learning as measured by functional MRI. \textit{Neuropsychologia, 38}, 1316-1324. \href{https://doi.org/10.1016/s0028-3932(00)00014-2}{DOI} \vspace{2mm}

Seger CA, Prabhakaran V, Poldrack RA, Gabrieli JD.  (2000). Neural activity differs between explicit and implicit learning of artificial grammar strings: An fMRI study. \textit{Psychobiology, 28}, 283-292. \href{https://doi.org/10.3758/bf03331987}{OA} \href{https://doi.org/10.3758/bf03331987}{DOI} \vspace{2mm}

Temple E, Poldrack RA, Protopapas A, Nagarajan S, Salz T, Tallal P, Merzenich MM, Gabrieli JD.  (2000). Disruption of the neural response to rapid acoustic stimuli in dyslexia: Evidence from functional MRI. \textit{Proceedings of the National Academy of Sciences of the United States of America, 97}, 13907-13912. \href{https://www.ncbi.nlm.nih.gov/pmc/articles/PMC17674}{OA} \href{https://doi.org/10.1073/pnas.240461697}{DOI} \vspace{2mm}

\subsection*{1999}Demb JB, Poldrack RA, Gabrieli JDE.  (1999). Functional neuroimaging of word processing in normal and dyslexic readers. In \textit{Converging Methods for Understanding Reading and Dyslexia }, 243-304. \vspace{2mm}

Illes J, Francis WS, Desmond JE, Gabrieli JD, Glover GH, Poldrack R, Lee CJ, Wagner AD.  (1999). Convergent cortical representation of semantic processing in bilinguals. \textit{Brain and Language, 70}, 347-363. \href{https://doi.org/10.1006/brln.1999.2186}{DOI} \vspace{2mm}

Poldrack RA, Prabhakaran V, Seger CA, Gabrieli JD.  (1999). Striatal activation during acquisition of a cognitive skill. \textit{Neuropsychology, 13}, 564-574. \href{https://doi.org/10.1037/0894-4105.13.4.564}{DOI} \vspace{2mm}

Poldrack RA, Wagner AD, Prull MW, Desmond JE, Glover GH, Gabrieli JD.  (1999). Functional specialization for semantic and phonological processing in the left inferior prefrontal cortex. \textit{NeuroImage, 10}, 15-35. \href{https://doi.org/10.1006/nimg.1999.0441}{DOI} \vspace{2mm}

Poldrack RA, Selco SL, Field JE, Cohen NJ.  (1999). The Relationship between Skill Learning and Repetition Priming: Experimental and Computational Analyses. \textit{Journal of Experimental Psychology: Learning Memory and Cognition, 25}, 208-235. \href{https://doi.org/10.1037/0278-7393.25.1.208}{DOI} \vspace{2mm}

\subsection*{1998}Gabrieli JD, Poldrack RA, Desmond JE.  (1998). The role of left prefrontal cortex in language and memory. \textit{Proceedings of the National Academy of Sciences of the United States of America, 95}, 906-913. \href{https://www.ncbi.nlm.nih.gov/pmc/articles/PMC33815}{OA} \href{https://doi.org/10.1073/pnas.95.3.906}{DOI} \vspace{2mm}

Gabrieli JD, Brewer JB, Poldrack RA.  (1998). Images of medial temporal lobe functions in human learning and memory. \textit{Neurobiology of Learning and Memory, 70}, 275-283. \href{https://doi.org/10.1006/nlme.1998.3853}{DOI} \vspace{2mm}

Poldrack RA, Gabrieli JD.  (1998). Memory and the brain: What's right and what's left?. \textit{Cell, 93}, 1091-1093. \href{https://doi.org/10.1016/s0092-8674(00)81451-8}{OA} \href{https://doi.org/10.1016/s0092-8674(00)81451-8}{DOI} \vspace{2mm}

Poldrack RA, Desmond JE, Glover GH, Gabrieli JD.  (1998). The neural basis of visual skill learning: An fMRI study of mirror reading. \textit{Cerebral Cortex, 8}, 1-10. \href{https://doi.org/10.1093/cercor/8.1.1}{OA} \href{https://doi.org/10.1093/cercor/8.1.1}{DOI} \vspace{2mm}

Poldrack RA, Logan GD.  (1998). What is the mechanism for fluency in successive recognition?. \textit{Acta Psychologica, 98}, 167-181. \href{https://doi.org/10.1016/s0001-6918(97)00041-3}{DOI} \vspace{2mm}

Wagner AD, Poldrack RA, Eldridge LL, Desmond JE, Glover GH, Gabrieli JD.  (1998). Material-specific lateralization of prefrontal activation during episodic encoding and retrieval. \textit{NeuroReport, 9}, 3711-3717. \href{https://doi.org/10.1097/00001756-199811160-00026}{DOI} \vspace{2mm}

\subsection*{1997}Cohen NJ, Poldrack RA, Eichenbaum H.  (1997). Memory for Items and Memory for Relations in the Procedural/Declarative Memory Framework. \textit{Memory, 5}, 131-178. \href{https://doi.org/10.1080/741941149}{DOI} \vspace{2mm}

Gabrieli JD, Keane MM, Zarella MM, Poldrack RA.  (1997). Preservation of implicit memory for new associations in global amnesia. \textit{Psychological Science, 8}, 326-329. \href{https://doi.org/10.1111/j.1467-9280.1997.tb00447.x}{DOI} \vspace{2mm}

Poldrack RA, Gabrieli JD.  (1997). Functional anatomy of long-term memory. \textit{Journal of Clinical Neurophysiology, 14}, 294-310. \href{https://doi.org/10.1097/00004691-199707000-00003}{DOI} \vspace{2mm}

Poldrack RA, Logan GD.  (1997). Fluency and response speed in recognition judgments. \textit{Memory and Cognition, 25}, 1-10. \href{https://doi.org/10.3758/bf03197280}{OA} \href{https://doi.org/10.3758/bf03197280}{DOI} \vspace{2mm}

Poldrack RA, Cohen NJ.  (1997). Priming of new associations in reading time: What is learned?. \textit{Psychonomic Bulletin and Review, 4}, 398-402. \href{https://doi.org/10.3758/bf03210800}{OA} \href{https://doi.org/10.3758/bf03210800}{DOI} \vspace{2mm}

\subsection*{1996}Poldrack RA.  (1996). On testing for stochastic dissociations. \textit{Psychonomic Bulletin and Review, 3}, 434-448. \href{https://doi.org/10.3758/bf03214547}{OA} \href{https://doi.org/10.3758/bf03214547}{DOI} \vspace{2mm}

\subsection*{1994}Poldrack RA, Cohen NJ.  (1994). On the representational/computational properties of multiple memory systems. \textit{Behavioral and Brain Sciences, 17}, 416-417. \href{https://doi.org/10.1017/s0140525x00035275}{DOI} \vspace{2mm}


\section*{Conference Presentations}
\noindent
\subsection*{2022}\textit{Toward an Open Science Ecosystem in Neuroimaging.}  Talk presented to the MRI Together meeting (virtual), December.

\textit{Toward an Open Science Ecosystem in Neuroimaging.}  Talk presented to the Ottawa Data Champions Workshop (virtual), November.

\textit{Applying fMRI in the domain of computer programming:  Promise and Pitfalls.} Talks presented to the Dagstuhl Seminar: Foundations for a New Perspective of Understanding Programming (virtual), October.

\textit{Reproducibility in fMRI: What's the problem?} Talk presented at the Neurohackademy, Seattle, WA, July.

\textit{Open Tools for Reproducible Science.} Talk presented to the AMP-SCZ Investigators Meeting (virtual), June.

\textit{Implementing open and reproducible science: Lessons learned from neuroimaging.} Talk presented to the NINDS workshop on Catalyzing Communities of Research Rigor Champions (virtual), May.

\textit{Making cognitive neuroscience more open and reproducible.} Talk presented to the Brain Space Initiative (virtual), January.

\textit{(How) can neuroimaging inform the architecture of the mind?} Dutch Distinguished Lecture Series in Philosophy and Neuroscience (virtual), January.

\subsection*{2021}\textit{Towards a more reproducible neuroscience.} Talk presented to the Annual Neuroscience Ireland meeting (virtual), September.

\textit{Towards a culture of computational reproducibility.} Keynote address to the Psychologie und Gehirn meeting (virtual), September.

\textit{Open infrastructure for reproducible science.} Talk presented to Amazon Web Services (virtual), May.

\textit{Some critical comments on data sharing (from a data sharing zealot).} Talk presented to the Vision Sciences Society Annual Meeting (virtual), May.

\textit{A close look at conceptual structure.} Talk presented at the NASEM Workshop on Behavioral Ontologies (virtual), May.

\textit{What do we want from a cognitive ontology?} Talk presented at the International Symposium on Cognitive Ontologies (virtual)., May.

\textit{General challenges for data sharing: Perspectives from a researcher/repository owner.} Talk presented at the NASEM Workshop on Changing the Culture of Data Management and Sharing (virtual), April.

\subsection*{2020}\textit{The OpenNeuro data repository.} Talk presented to the FABBS Webinar on Data Depositories (virtual), November.

\textit{Cognitive ontologies: From Top to Bottom.} Talk presented at The Problem of Cognitive Ontology: Implications for Scientific Knowledge, Pittsburgh (virtual), September.

\textit{Best practices for reproducible neuroimaging.} Talk presented at the FLUX Conference (virtual), September.

\textit{What's wrong with neuroimaging research, and how can we make it right?} Talk presented at the Neurohackademy, Seattle, WA (virtual), July.

\textit{The measurement crisis in cognitive neuroscience.} Talk presented at the Reading Emotions Conference, Reading, UK (virtual), June.

\textit{Reproducibility in neuroimaging: What's the problem?} Talk presented to the OHBM Australia Chapter (virtual)., May.

\textit{Towards an open science ecosystem for neuroimaging.} Keynote talk presented to the Neuromatch.io conference (virtual)., March.

\textit{Toward an ecosystem for open and reproducible science.} Talk presented to the Max Planck Cognition Academy (virtual)., January.

\subsection*{2019}\textit{Building around standards: The role of BIDS in neuroimaging data sharing.} Talk presented at the ACNN Annual Meeting, Ann Arbor, MI, September.

\textit{What's wrong with neuroimaging research, and how can we make it right.} Talk presented at the Neurohackademy, Seattle, WA, June.

\textit{OpenNeuro: Data sharing for the BRAIN Initiative.} Talk presented at the OHBM Open Science SIG, Rome, June.

\textit{The importance of standards for data sharing in neuroimaging.} Keynote presented at the Human Connectome Project Investigators Meeting, Bethesda, MD, May.

\textit{Grand views and potholes on the road to precision neuroimaging.} Talk presented at the Neuroimaging and Modulation in Obesity and Diabetes Research 10th Anniversary Meeting, Bethesday, MD, April.

\textit{OpenNeuro: An open archive for BRAIN Initiative neuroimaging data.} Talk presented at the BRAIN Initiative Investigators Meeting, Washington, DC, April.

\textit{Towards an open science ecosystem for neuroimaging.} Talk presented to the Montreal Neurological Institute Open Science Symposium (virtual)., March.

\subsection*{2018}\textit{Toward a Computational Ontology of Mental Function.} Keynote presented at the Deliberations on Cognitive Ontology Conference, St. Louis, MO, October.

\textit{How Data Sharing Succeeded in Neuroimaging.} Keynote presented at the SRCD DEVSEC2018 Meeting, Phoenix, AZ, October.

\textit{Towards a robust data organization scheme for neuroimaging: the Brain Imaging Data Structure.} Keynote presented at Neuroinformatics 2018, Montreal, August.

\textit{Reproducibility in neuroimaging: What is the problem.} Talk presented at Neurohackademy 2018, Seattle, July.

\textit{What is the structure of self-regulation.} Talk presented at the Association for Psychological Science Annual Meeting, San Francisco, May.

\textit{From Big Data to Big Knowledge. Analytic and conceptual challenges for developmental neuroimaging.} Talk presented at the FLUX Satellite meeting, Chapel Hill, NC, May.

\textit{Creating a reproducible research pipeline.} Talk presented at the BBSRC STARS Course on Advanced Methods for Reproducible Science, Windsor, UK, April.

\subsection*{2017}\textit{Making neuroimaging more reproducible and transparent.} Talk presented at the Annual Meeting of the American Academy of Adolescent and Child Psychiatry, Washington, DC, October.

\textit{Reproducibility in neuroimaging: Challenges and solutions.} Kavli Workshop Lecture, Society for Neuroeconomics, Toronto, October.

\textit{Reproducibility in neuroimaging: What's the problem.} Talk presented at NeuroHackWeek 2017, Seattle, September.

\textit{Computational infrastructure for cognitive neuroscience: The Poldracklab experience.} Talk presented remotely to NeuroComp17, Madison, WI, August.

\textit{Making neuroimaging more reproducible and transparent.} Talk presented at the International Conference on Cognitive Neuroscience (ICON), Amsterdam, August.

\textit{Improving reproducibility of fMRI studies.} Talk presented at the Society for the Study of Ingestive Behavior, Montreal, July.

\textit{The dynamics of human brain function.} Talk presented at the Neuroscience Workshop Saclay, Paris, June.

\textit{Building reproducible analysis workflows.} Talk presented at the BBSRC Workshop on Advanced Methods for Reproducible Science. Windsor, UK, March.

\textit{Improving the reproducibility of computational research: Cyberinfrastructure for cognitive neuroscience.} Distinguished Lecture, National Science Foundation, Directorate for Computer and Information Science and Engineering., March.

\textit{Mechanisms of behavioral change.} Talk presented at the Geneva-Princeton Learning Meeting, January.

\subsection*{2016}\textit{ Improving the reproducibility of neuroimaging research.} Talk presented at Neurohackweek, Seattle, September.

\textit{The future of fMRI in cognitive neuroscience.} Talk presented at the UCLA Advanced Neuroimaging Summer School, July.

\textit{Inferring mental states from imaging data: OpenfMRI and the Cognitive Atlas.} Talk presented at the Organization for Human Brain Mapping Annual Meeting, Geneva, June.

\textit{Automatic influences on value.} Talk presented at Decision Neuroscience Annual Meeting, Philadelphia, June.

\textit{Representing Knowledge in Psychology: Challenges and Perspectives.} Talk presented at Association for Psychological Science Annual Meeting, Chicago, May.

\textit{Changing choices and preferences through automatic mechanisms.} Talk presented at Association for Psychological Science Annual Meeting, Chicago, May.

\textit{Using neuroscience to refine the ontology of psychology.} Talk presented at Rethinking the Taxonomy of Psychology, London, Ontario, April.

\subsection*{2015}\textit{Cognitive ontologies, data sharing, and reproducibility..} Talk presented at the Dagstuhl Perspectives Workshop on Digital Scholarship and Open Science in Psychology and the Behavioral Sciences, Dagstuhl, Germany, July.

\textit{The future of fMRI in cognitive neuroscience.} Talk presented at the UCLA Advanced Neuroimaging Summer School, June.

\textit{Peripheral gene expression and brain function.} Talk presented at the Organization for Human Brain Mapping Annual Meeting, Honolulu, June.

\textit{Decision making and cognitive control: Towards a new synthesis.} Talk presented at the Nutritional Neuroscience Symposium, Utrecht, Netherlands, March.

\subsection*{2014}\textit{Toward an ecosystem for task fMRI data sharing: Neurosynth, Neurovault, and OpenfMRI.} Talk presented at the American Academy of Child and Adolescent Psychiatry, San Diego, October.

\textit{Using neuroimaging to infer mental states: A guided tour through the minefield.} Kavli Foundation Workshop, Society for Neuroeconomics Annual Meeting, Miami, September.

\textit{Towards a personalized cognitive neuroscience: The MyConnectome Project.} Keynote address presented at the International Conference on Cognitive Neuroscience (ICON), Brisbane, July.

\textit{Large-scale decoding of neurocognitive organization.} Keynote address presented at the Pattern Recognition in Neuroimaging meeting, Tubingen, Germany, June.

\textit{The Cognitive Atlas Project.} Talk presented at the Annual Meeting of the Association for Psychological Science, San Francisco, May.

\subsection*{2013}\textit{Is ``efficiency'' a useless concept.} Talk presented at the First Flux Congress, Pittsburgh, September.

\textit{Beyond the GLM: Advanced fMRI analysis techniques.} Talk presented at the QBIN Summer School on Basic and Advanced fMRI, August.

\textit{The Cognitive Atlas Project.} Keynote address presented at the Cognitive Ontologies meeting, Manchester, June.

\textit{From neuroimaging to mental structure.} Keynote address presented at the Organization for Human Brain Mapping Annual Meeting, Seattle, June.

\subsection*{2012}\textit{Cognitive Neuroinformatics.} Keynote address presented at Neuroinformatics 2012, Munich, September.

\textit{The future of fMRI in the cognitive neuroscience of language.} Talk presented at the International Workshop on Brain, Cognition, and Learning, Beijing, June.

\subsection*{2011}\textit{Learning and changing habits.} Talk presented at the International Workshop on Brain, Cognition, and Learning, Beijing, May.

\subsection*{2010}\textit{Toward a semantic infrastructure for cognitive neuroscience: The Cognitive Atlas.} Talk presented at the Cognitive Neuroscience Society Annual Meeting, Montreal, April.

\textit{The Cognitive Atlas: employing interaction design processes to facilitate collaborative ontology creation.} Talk presented to the HCLS 2010 Meeting on The Future of the Web for Collaborative Science, Raleigh, NC, April.

\textit{Inference and Imaging.} Talk presented at the Hastings Center Workshop on Interpreting Neuroimages II, Philadelphia, February.

\subsection*{2009}\textit{Reading mental states from neuroimaging data: From reverse inference to pattern classification.} Talk presented at the UC Irvine Institute for Mathematical Behavioral Sciences workshop on Inference and Imaging, November.

\textit{Reading mental states from neuroimaging data: From reverse inference to pattern classification.} Keynote lecture presented at Conceptual Issues in fMRI Interpretation: An Interdisciplinary Workshop, Guelph, Ontario, May.

\textit{What can we predict from fMRI. Lessons for developmental neuroimaging.} Talk presented at the Conference on Methods and Challenges in Developmental Neuroimaging, Amsterdam, May.

\subsection*{2008}\textit{Beyond Phrenology: Neuroimaging and cognitive ontologies.} Talk presented at the Annual Meeting of the Organization for Human Brain Mapping, Melbourne, Australia, June.

\textit{Neural systems for learning and controlling skills.} Talk presented at the Georgetown CBBC Workshop on The Neurocognition of Language and Memory: Retention, Attrition, and Aging, March.

\textit{Prospect theory and the brain.} Talk presented at the NYU Neuroeconomics conference, January.

\subsection*{2007}\textit{Distinguishing category versus response learning using a reversal paradigm..} Talk presented at the Memory Disorders Research Society Annual Meeting, Cambridge, England, September.

\textit{Memory systems and category learning.} Talk presented at the European Conference on Visual Perception, Arezzo, Italy, August.

\textit{What can (and can\~Ot) fMRI tell us about the brain.} Talk presented at the California Science and the Law Institute Conference on Neurobiology and the Courts. Riverside, CA, March.

\subsection*{2006}\textit{Balancing risk and reward in decision making: An fmri study of the Balloon Analog Risk Task.} Talk presented at the Affect, Motivation, and Decision Making Conference, Ein Boqeq, Israel., December.

\textit{Skill acquisition, multiple memory systems, and cognitive control.} Talk presented at the Rodin Remediation Academy Annual Meeting, Georgetown University, October.

\textit{Learning-related changes during skill learning: Evidence for memory system interactions.} Talk presented at the at the Marburg Conference on Neuroimaging and  Theories of Memory, Marburg, Germany, July.

\textit{Imaging skill learning using fMRI: Insights and challenges.} Talk presented at the John Merck Fund Summer Institute on the Biology of Developmental Disabilities, Princeton, June.

\subsection*{2005}\textit{Methodological and conceptual challenges for developmental fMRI.} Talk presented at the Amsterdam Conference on Developmental Cognitive Neuroscience, Amsterdam, June.

\textit{The development of phonological awareness.} Talk presented at the Amsterdam Conference on Developmental Cognitive Neuroscience, Amsterdam, June.

\textit{The organization of cognitive functions in the left inferior prefrontal cortex.} Talk presented at the Cognitive Neuroscience Society Annual Meeting, New York, April.

\subsection*{2003}\textit{Interactive memory systems in humans.} Talk presented at the SFN Pre-Symposium on Interactive Memory Systems, New Orleans, November.

\textit{Neural systems for rapid naming and phonological awareness.} Talk presented at the International Dyslexia Association annual meeting, San Diego, January.

\subsection*{2002}\textit{Functional imaging of of classification learning.} Talk presented at the NIPS Workshop on Foundations and Modeling in Neuroimaging, Whistler, BC, December.

\textit{The neural basis of skill learning.} Talk presented at the Memory Disorders Research Society Annual Meeting, San Francisco, August.

\textit{ The role of the basal ganglia in category learning.} Talk presented at the First Cognitive Neuroscience of Category Learning Conference, New York, June.

\subsection*{2001}\textit{Imaging of perceptual and cognitive skill learning.} Talk presented at the Third Annual International Conference on Memory, Valencia, Spain, June.

\textit{The neural basis of reading and dyslexia: Evidence from magnetic resonance imaging.} Talk presented at the fourth Annual Japanese-American Frontiers in Science Symposium, Tokyo, Japan., March.

\textit{Reading and the brain: New directions for diagnosis and treatment.} Talk presented at the 17th Annual Learning Differences Conference, Cambridge, MA, January.

\subsection*{2000}\textit{fMRI studies of striatal and medial temporal lobe activation during category learning.} Talk presented at the Memory Disorders Research Society Annual Meeting, Toronto, August.

\textit{A neural basis for reading skill: Evidence from microstructural brain imaging.} Talk presented at the National Dyslexia Research Foundation Annual Conference, Elounda, Greece, June.

\textit{Brain Structure and Function: The Relation to Reading.} Talk presented at the BrainConnection.com Brain Research and Learning Conference, San Francisco, January.

\subsection*{1998}\textit{Multiple functional roles of the left inferior prefrontal cortex: Evidence from neuroimaging.} Talk presented at the Academy of Aphasia Annual Meeting, November 1998, Santa Fe, NM, April.


\section*{Invited addresses and colloquia (* - talks given virtually)}
\noindent
2023: University of Tubingen*

2022: University of Michigan*, Brigham and Womens Hospital*, University of Alabama at Birmingham*, Tehran University of Medical Sciences*

2021: University of Zurich*, Purdue University*, University of Florida*, King's College London*, University of Texas Rio Grande Valley*, SUNY Binghamton*, University of Connecticut*

2020: University of Colorado, Boulder*, University of Virginia, Biomedical Data Science Program*, Florida International University*, University Medical Center, Hamburg, Germany*, Max Planck Institute for Human Cognitive and Brain Sciences, Leipzig, Germany*,  UC Davis*

2019:  UCLA,  IMT Lucca (Italy),  Tel Aviv University,  Ecole Normale Superieure (Paris)*,  Johns Hopkins University (Biostatistics)*,   NIMH Machine Learning Group*,  Johns Hopkins University (ECE)* 

2018:  Penn State University,  Icahn School of Medicine at Mt. Sinai,  University of Wisconsin,  University of Miami. 

2017:  Caltech,  University of Pennsylvania,  University of Maryland,  Ghent University,  Harvard University,  National University, Singapore 

2016:  Wellcome Trust Center for Neuroscience, University College, London,  MRC Cognition and Brain Unit, Cambridge,  Duke University,  Dartmouth University,  Rutgers-Newark 

2015:  Tel Aviv University,  University Medical Center, Utrecht, Netherlands,  Montreal Neurological Institute,  Cornell University 

2014:  Laureate Insitute, Tulsa, OK,  Yale University,  Michigan State University 

2013:  Columbia University,  UT San Antonio,  Rotman Research Institute, Toronto,  University of Michigan,  UCLA,  NIMH,  NIDA,  Carnegie-Mellon University,  University of Oregon,  Mt. Sinai School of Medicine,  Washington University, St. Louis

2012:  University of Texas at Dallas,  Ohio State University,  Stanford University,  Tokyo Institute of Technology

2011:  University College London,  Cambridge University,  Oxford University,  Georgia Tech University,  University of Georgia,  Brown University,  Humboldt University (Berlin),  National Taiwan University (Taipei).

2010:  Beijing Normal University,  Duke-NUS School of Medicine, Singapore,  Princeton University,  UC Davis,  University of Maryland,  Auburn University,  Texas A\&M University 

2009:  University of Pennsylvania,  UC San Diego,  University of Texas at Austin,  Baylor College of Medicine

2008:  University of Vermont,  SUNY Stony Brook,  Vanderbilt University,  University of Illinois at Chicago,  University of Texas at Austin,  University of Missouri-Columbia,  Washington University-St. Louis,  Cal Tech,  Neurospin (Orsay, France),  Oxford University,  University College London

2007:  Duke University,  New York University,  University of Texas,  MIT,  University College London

2006:  Salk  Institute,  Johns Hopkins University,  Ben Gurion University (Beer Sheva, Israel)

2005:  University of Arizona,  Medical College of Wisconsin,  Washington University-St. Louis, Karolinska Institute, Lund University (Sweden), Danish Technical University

2004:  Rotman Research Institute, University of Toronto,  University of California, San Diego,  University of Colorado, Boulder,  Colorado State University,  University of Illinois at Urbana-Champaign

2003:  NIMH, Clinical Brain Disorders Branch,  Max Planck Institute for Cognitive Neuroscience, Leipzig, Germany,  University of California at Irvine, Center for Neurobiology of Learning and Memory,

2002:  Learning and the Brain conference, Cambridge, MA,  Los Alamos National Laboratory, Center for Nonlinear Studies

2001:  Institute for Cognitive Neuroscience, London,   Wellcome Department of Cognitive Neurology, London

2000:  University of Connecticut,  MIT,  Boston University,  Boston VA Medical Center

1999:  Center for Psychological Studies, Berkeley, CA ,  Harvard University

1997:  University of California at Berkeley 

1996:  University of California at Santa Cruz 

1994:  Rice University 


\begin{center}
{\footnotesize Last updated: \today — Generated using \href{https://github.com/poldrack/autoCV}{AutoCV} 
}
\end{center}


\end{document}
